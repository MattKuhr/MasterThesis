\section{Problemstellung und Requirements}
\label{sec-3}
\fbox{
	\parbox{\linewidth}{
		\textit{Ziel des Kapitels:}\\
		Problemstellung formulieren und eingrenzen, Anforderungen herausarbeiten.\\[6px]
		\textit{Inhalte:}
		\begin{itemize}
			\item Problem
			\item Einschränkungen
			\item Anforderungen
		\end{itemize}
		
		\textit{(Optional) Literatur:}	
		%\begin{itemize}
		%\end{itemize}
}}

\subsection{Anforderungen}
\label{sec-3-1}
Anforderungen an die Anwendung\\

\subsubsection{Was - Physik}

\textit{Darzustellende Informationen:}
\begin{itemize}
	\item Magnetfeld von Erde und Spule
	\begin{itemize}
		\item Stärke
		\item Richtung
		\item Homogenität
		\item Inhomogenität am Rand der Spule andeuten (Optional) 
	\end{itemize}
	\item Stromfluss durch die Spule
	\begin{itemize}
		\item Richtung
		\item Kennzeichnung von Plus und Minus
		\item Stärke (Optional) 
	\end{itemize}
	\item Kompass
	\begin{itemize}
		\setlength{\itemsep}{-0.25em}
		\item Nordrichtung
		\item Grobe Auslenkung der Nadel
	\end{itemize}
	\item Weitere Informationen (Optional)
	\begin{itemize}
		\item Windungszahl der Spule
		\item Durchmesser und Abstand der Spulen
		\item Numerische Werte und Informationen (z.B. Fließt aktuell Strom, angelegte Stromstärke, angenommene Stärke des Erdmagnetfeldes, systematischer und zufälliger Fehler, etc.)
	\end{itemize}
\end{itemize}

\textit{Anforderungen:}
\begin{itemize}
	\item Darstellungen müssen physikalisch korrekt und interpretierbar sein
	\item Nutzer darf nicht in seiner Interaktion mit dem Versuchsaufbau und relevanten Materialien eingeschränkt werden
	\item Sicherheitsrelevante Aspekte beachten
\end{itemize}

We suggest developing MR applications that enable learners to actually see and understand the fundamental facts. Specialized sensors can measure environmental data or the current status of an experiment. Voltage and current could be directly displayed within the wires during electrical engineering classes (Beheshti et al., 2017) or heat propagation in metals within physic classes (M. P. Strzys et al., 2018). MR displays in combination with sensors allows us to extend the human vision and visualize in-depth details of learning material in place of occurrence.

Once deployed, MR systems can overcome this limitation and enhance learning. In addition, interactive learning and exploring environments can be created that extend the current body of learning material. We envision interactive experiments that were not possible to realize before because of time, financial, or security constraints. For example, chemistry students could safely explore chemical reactions with hazardous elements or biology students can examine samples under an augmented micro- scope that are usually not accessible.

\subsubsection{Wie - Technische Seite}

Technische Randbedingungen durch die Brille im Zusammenhang mit Anwendungsfall
\begin{itemize}
	\item Größe, Geschwindigkeit, Farbe, Distanz zur Kamera von Objekten
	\item Zusammenspiel der Darstellungen mit der Umgebung beachten
	\item Stabilität der Hologramme gewährleisten
	\item 60 FPS stabil halten, stark spiegelnde oder transparente Oberflächen vermeiden, mögliche Einflüsse auf die Sensoren beachten
	\item Usability und UX Empfehlungen beachten
\end{itemize}

\subsection{Problemstellung}
\label{sec-3-2}
Zwei Aspekte: Anforderungen von der physikalischen und der technischen Seite

Problem: Anforderungen und technische Möglichkeiten sowie Einschränkungen zusammenbringen und eine Lösung entwickeln
\begin{itemize}
	\item Was soll dargestellt werden? (Was soll nicht dargestellt werden?)
	\item Wie soll es dargestellt werden?
	\item Wie soll damit interagiert werden?
\end{itemize}

