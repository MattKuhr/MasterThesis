\section{Problemstellung und Requirements}
\label{sec-3}
\fbox{
	\parbox{\linewidth}{
		\textit{Ziel des Kapitels:}\\
		Problemstellung formulieren und eingrenzen, Anforderungen herausarbeiten.\\[6px]
		\textit{Inhalte:}
		\begin{itemize}
			\item Problem
			\item Einschränkungen
			\item Anforderungen
		\end{itemize}
		
		\textit{(Optional) Literatur:}	
		%\begin{itemize}
		%\end{itemize}
}}

\subsection{Anforderungen}
\label{sec-3-1}
Anforderungen an die Anwendung\\

\textit{Darzustellende Informationen:}
\begin{itemize}
	\item Magnetfeld von Erde und Spule
	\subitem Stärke
	\subitem Richtung
	\subitem Homogenität
	\subitem Inhomogenität am Rand der Spule andeuten (Optional)
	\item Stromfluss durch die Spule
	\subitem Richtung
	\subitem Kennzeichnung von Plus und Minus
	\subitem Stärke (Optional)
	\item Kompass
	\item Weitere Informationen zum Versuchsaufbau
\end{itemize}

\textit{Anforderungen:}
\begin{itemize}
	\item Darstellungen müssen physikalisch korrekt und interpretierbar sein
	\item Nutzer darf nicht in seiner Interaktion mit dem Versuchsaufbau und relevanten Materialien eingeschränkt werden
	\item Sicherheitsrelevante Aspekte beachten
\end{itemize}

Technische Randbedingungen durch die Brille im Zusammenhang mit Anwendungsfall
\begin{itemize}
	\item Distanz, Größe und Geschwindigkeit der virtuellen Objekte, passt hier aufgrund Größe der Spulen
	\item Keine Störung der Brillen-Hardware (z.B. durch starkes Magnetfeld)
	\item Performance Limitierung, also z.B. keine Echtzeit-Berechnung komplexer Magnetfelder
	\item FoV Limitierung muss beachtet werden
	\item Clutter
	\item Transparenz hängt von Farbe und Hintergrund ab
	\item ....
\end{itemize}

\subsection{Problemstellung}
\label{sec-3-2}
Zwei Aspekte: Anforderungen von der physikalischen und der technischen Seite

Problem: Anforderungen und technische Möglichkeiten sowie Einschränkungen zusammenbringen und eine Lösung entwickeln
\begin{itemize}
	\item Was soll dargestellt werden? (Was soll nicht dargestellt werden?)
	\item Wie soll es dargestellt werden?
	\item Wie soll damit interagiert werden?
\end{itemize}

