\section{Anforderungen und Problemstellung}
\label{sec-3}
Ziel dieser Arbeit ist es, am konkreten Beispiel des in Kap. \ref{sec-3-2} beschriebenen Laborversuches zu untersuchen, wie die HoloLens in der Physik Lehre eingesetzt werden kann. Die dazu notwendigen Hintergrundinformationen von technischer und physikalischer Seite wurden im vorangegangenen Kapitel erläutert. Außerdem wurden bestehende Lösungen vorgestellt und in Zusammenhang mit der Zielsetzung dieser Arbeit gebracht. Damit sind die Grundlagen geschaffen, um konkrete Anforderungen und Probleme zu formulieren.

\subsection{Anforderungen}
\label{sec-3-1}
Die Anforderungen an eine Umsetzung teilen sich in zwei Bereiche: Inhaltliche Anforderungen aus Sicht der Physik und technische Bedingungen von Seiten der HoloLens. Ein Design muss beide berücksichtigen und zu einer Lösung zusammenführen. Dabei bestimmt das Anwendungsszenario vor allem die Frage, was dargestellt werden soll, während die technischen Gegebenheiten vorwiegend auf das ''Wie'' der Umsetzung Einfluss nehmen. Im Weiteren sollen die Anforderungen beider Seiten formuliert werden.

\subsubsection{Physikalische Seite}
\label{sec-3-1-1}
Von Seiten der Physik teilen sich die Anforderungen nochmals in zwei Bereiche. Zum einen sind inhaltliche Kriterien an die Anwendung zu stellen. Der Versuch dient dazu, Lernenden bestimmte physikalische Konzepte und Zusammenhänge zu vermitteln. An diesen muss sich eine Lösung orientieren und ableiten, welche Informationen mit der HoloLens darzustellen sind.\\
\noindent\hspace*{5mm}
Zum anderen bringt der Anwendungsfall auch Einschränkungen in Bezug darauf mit sich, wie die Darstellungen erfolgen sollen. Letztere müssen unter anderem für den Anwender hinsichtlich ihrer physikalischen Bedeutung interpretierbar sein und dürfen keine falschen Eindrücke erwecken.\\

\textit{Relevante physikalische Zusammenhänge}\\
Der Fokus dieser Arbeit liegt darin, zusätzliche Informationen zu physikalischen Zuständen anzuzeigen. Eine Unterstützung beim Aufbau oder der Durchführung des Versuches ist nicht vorgesehen. Die Anwendung zielt zunächst nur auf die Bestimmung der Stärke des Erdfeldes ab. Die in Kap. \ref{sec-2-3-3} genannten ersten beiden Schritte der Durchführung (Ausrichtung von Nadel und Spule) werden also vorausgesetzt. Ausgangspunkt dafür, welche Informationen zu visualisieren sind, stellen die zu vermittelnden Zusammenhänge dar. In Zusammenarbeit mit Experten wurden die folgenden Fragen als zentral für das Verständnis des Versuches festgehalten \cite{Reinholz18}.

\begin{itemize}
	\setlength{\itemsep}{-3pt}
	\item Welche Magnetfelder entstehen und welche Eigenschaften weisen sie auf?
	\item Wie hängen Ausrichtung des Magnetfeldes der Spule und Stromfluss zusammen?
	\item Wie verändert sich die Feldstärke der Spule bei einem sich ändernden Stromfluss?
	\item Wann ist die Nadel um 45° ausgelenkt?
\end{itemize}
% QUELLE H. Reinholz, A. Sengebusch, R. Leppin, Rostock, 03.12.2018, Persönliches Gespräch 

Das Verständnis dieser Sachverhalte soll im Rahmen der zu erarbeitenden Lösung durch zusätzlich angezeigte Informationen gestützt werden. Das Ziel ist dabei nicht die eigenständige Beantwortung dieser Fragen, sondern vielmehr die Bereitstellung von dafür wichtigen Informationen.\\

\textit{Anforderungen an die Umsetzung}\\
Neben den Voraussetzungen bezüglich dessen, was dargestellt wird, sind auch Anforderungen daran zu stellen, wie dies erfolgt. Ziel des Versuches ist ein qualitatives Verständnis der zuvor genannten Eigenschaften und Zusammenhänge. Darstellungen sollten also auf in erster Linie einen qualitativen Eindruck vermitteln.\\

Damit die dargestellten physikalischen Zustände auch als solche interpretierbar sind, sollte auf etablierte Modelle zurückgegriffen werden. Andernfalls müsste ein Nutzer neben den eigentlichen Sachverhalten zunächst die Bedeutung der virtuellen Elemente verstehen. Darüber hinaus dürfen durch die angezeigten Objekte keine falschen Eindrücke entstehen, da diese dem eigentlichen Ziel entgegen stünden.\\
\noindent\hspace*{5mm}
Sofern Zusammenhänge direkt repräsentiert werden, sollen diese physikalisch korrekt wiedergegeben werden. Um außerdem eine natürliche und sichere Nutzung der Gerätschaften zu gewährleisten, soll der Nutzer nicht in seiner Interaktion mit dem Versuchsaufbau und relevanten Materialien eingeschränkt werden.\\
\noindent\hspace*{5mm}
Um den Einsatz eines Prototypen auch für Nutzer ohne Erfahrung mit der HoloLens zu ermöglichen, sollte die Lösung keine Vorkenntnisse zu dem Gerät voraussetzen. Um schließlich die praktische Umsetzung zu gewährleisten, sollen die vorhandenen Gerätschaften und Lehrmittel genutzt werden.\\

Diese aufgestellten Rahmenbedingungen fasst Tabelle \ref{tab:req} noch einmal zusammen. Nachdem die inhaltlichen Anforderung aufgestellt wurden, sollen nun die technischen folgen.

\bgroup
\setlength\extrarowheight{0pt}
\def\arraystretch{1.25}
\begin{table}[h!]
	\centering
	\begin{tabular}{l|l}
		Bereich & Anforderung\\
		\hline
		\hline
		Darstellungen sollen.. & ..physikalische Eigenschaften qualitativ wiedergeben\\
		 & ..physikalisch interpretierbar sein\\
		 & ..und falsche physikalische Eindrücke vermeiden\\
		 \hline
		Anwendung soll.. & ..die Interaktion mit den Gerätschaften nicht behindern\\
		 & ..nutzbar sein für unerfahrene Nutzer \\
		 & ..und vorhandene Lehrmittel verwenden
	\end{tabular}\caption{\label{tab:req} Inhaltliche Anforderungen an die Art und Weise einer Umsetzung.}
\end{table}
\egroup

\subsubsection{Technische Seite}
\label{sec-3-1-2}
In Kapitel \ref{sec-2-1} wurde die HoloLens mit ihren technischen Möglichkeiten aber auch Einschränkungen vorgestellt sowie Auswirkungen auf das Anwendungsdesign aufgezeigt. Die genannten Probleme sollen möglichst vermieden werden, indem die Anwendung an die speziellen Gegebenheiten der HoloLens angepasst wird. Dementsprechend sollen auch von technischer Seite Anforderungen aufgestellt werden.\\



\subsection{Präzisierung der Problemstellung}
\label{sec-3-2}
Die inhaltlichen und technischen Anforderungen stecken den Rahmen ab, für den eine Lösung zu erarbeiten ist. Dazu ist ein Design nötig, dass die inhaltlichen Anforderungen bedient, dabei die Möglichkeiten der HoloLens nutzt und gleichzeitig um technische Einschränkungen und Besonderheiten herum navigiert. Um die Fragestellung zu beantworten, wie die HoloLens am konkreten Beispiel eingesetzt werden kann, sind drei zentrale Fragen zu adressieren:
\begin{itemize}
	\setlength{\itemsep}{-5pt}
	\item Was soll mit der HoloLens dargestellt werden?
	\item Wie soll diese Darstellung erfolgen?
	\item Und wie soll mit den dargestellten Informationen interagiert werden?
\end{itemize}

\textbf{\textit{Was soll dargestellt werden?}}\\
Zunächst stellt sich die Frage, welche Objekte die HoloLens zusätzlich anzeigen soll. Aus den genannten physikalischen Zusammenhängen sind in Zusammenarbeit mit Experten wichtige Informationen abzuleiten, die repräsentiert werden sollen. Gegebenenfalls ist auch zu entscheiden, ob eine Darstellung durch die HoloLens erfolgen, oder die Information durch ein reales Objekt vermittelt werden soll. In letzterem Fall ist darauf zu achten, dass die Objekte durch virtuelle Darstellungen nur bedingt überblendet werden sollten.\\
\noindent\hspace*{5mm}
Im Ergebnis soll eine Liste von durch die Anwendung dargestellten Informationen aufgestellt werden.\\

\textbf{\textit{Wie soll diese Darstellung erfolgen?}}\\
Die wohl umfassendste Frage besteht darin, wie die Informationen und Zusammenhänge visualisiert werden sollen. Für die erarbeiteten Elemente ist ein visuelles Design mit einer technischen Umsetzung zu erarbeiten. Was soll textuell, als 2D oder als 3D Geometrie mit der HoloLens dargestellt werden? Welche Objekte werden gewählt und wie angeordnet? Wie erfolgt die technische Realisierung?\\
\noindent\hspace*{5mm}
Daraus resultiert eine Umsetzung für die anzuzeigenden Objekte, die im Ergebnis anhand der formulierten Anforderungen an die Umsetzung bewertet wird.\\

\textbf{\textit{Wie soll damit interagiert werden?}}\\
Nicht zuletzt ist ein Konzept für die Interaktion zu erstellen. Welche Eingabemethoden werden dem Nutzer angeboten und wie wird der Nutzer über diese informiert? Welche Aktionen soll die Anwendung zur Verfügung stellen?\\
\noindent\hspace*{5mm}
Das Ergebnis ist ein Konzept für die Interaktion und den Ablauf der Applikation.
