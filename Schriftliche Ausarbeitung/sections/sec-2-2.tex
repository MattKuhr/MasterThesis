\subsection{Mixed Reality in der Lehre}
\label{sec-2-2}
\fbox{
\parbox{\linewidth}{
	\textit{Ziel des Kapitels:}\\
	State of the Art in der Überschneidung mit dem Education Bereich vorstellen.\\[6px]
	\textit{Inhalte:}
	\begin{itemize}
		\item Aktueller Einsatz von MR, insb. der HoloLens, in der Physik
		\item Einsatz von MR in Lehre allgemein, nur relevante Aspekte
	\end{itemize}
	
	\textit{Wichtige Literatur:}	
	\begin{itemize}
		\item \cite{Buchau09}
		\item Experimenting with electromagnetism using augmented reality: Impact on flow student experience and educational effectiveness Ibanez \cite{Ibanez14}
		\item Influences of AR-Supported Simulation on Learning Effectiveness in Face-to-face Collaborative Learning for Physics \cite{Li11}
		\item Real Time Simulation Method of Magnetic Field for Visualization System With Augmented Reality Technology \cite{Matsutomo13}
		\item Physics holo.lab learning experience: using smartglasses for augmented reality labwork to foster the concepts of heat conduction \cite{Strzys18}
		\item Using Augmented Reality for Teaching Physics \cite{Techakosit15}
		\item HolOsci: Hololens Augmented Reality Oscilloscope Based Support for Debugging Electronics Circuits \cite{Javaheri18}
		\item PhET: Simulations that enhance learning \cite{Wieman08}
	\end{itemize}
}}\\

Ausbildung und Training sind ein wichtiges Anwendungsgebiet für Mixed Reality Technologie. Das gilt besonders für den Augmented Reality Bereich. In der Literatur findet sich ein breites Spektrum an Arbeiten zum Einsatz von AR zur Vermittlung von Lerninhalten und zum Training. Einen Überblick über die Forschung im Einsatz von AR in der Lehre geben unter anderem die Arbeiten von Bacca et. al. und Chen et. al. \cite{Chen2017, Bacca14}.\\

\begin{itemize}
	\item Bildung ist wichtiges Anwendungsgebiet für MR im Allgemeinen und AR im Besonderen
	\item Ergänzung des realen ohne völlige Abschottung im Fall von AR
	\item Direkte Verknüpfung (räumlich-zeitlich) für Lerneffekt vermutlich positiv \cite{Akcayir17}, \cite{Chen2017}, \cite{Knierim18}
	\item breitere Studien zu positivem Effekt von AR
\end{itemize}

\subsubsection{In der Physik}
\begin{itemize}
	\item AR gut geeignet für Elektromagnetismus, Darstellung von M und EM Feldern
	\item Mehrere empirische eval in der Physik, die positive Effekte zeigen
	\item mehrere aktuelle beispiele mit der HoloLens
\end{itemize}
Buchau \cite{Buchau09}, Strzys \cite{Strzys18}, Javaheri \cite{Javaheri18}, Techakosti \cite{Techakosit15} vorstellen. Weitere Arbeiten erwähnen.\\
Einordnung in Virtual Continuum und Anwendungsgebiet in der Physik.
\begin{itemize}
	\item Buchau links im AR Bereich, Magnetismus und Elektromagnetismus im Anwendungsbereich
	\item Strzys weiter rechts im AR Bereich, Thermodynamik im Anwendungsbereich
	\item Javaheri mittig im AR Bereich, Elektronik und Schaltungen im Anwendungsbereich
	\item ...
	\item ...
\end{itemize}

\subsubsection{In anderen Bereichen}
\begin{itemize}
	\item Beispiel HoloMuse, Insight Heart, Galaxy Explorer 
\end{itemize}

Fazit: So und so wird MR und HoloLens aktuell eingesetzt

