\section{Ergebnisse und Diskussion}
\label{sec-6}
\fbox{
\parbox{\linewidth}{
	\textit{Ziel des Kapitels:}\\
	Ergebnisse der Umsetzung vorstellen und auf die Fragestellung anwenden.
}}
Ziel dieser Arbeit war es, zu untersuchen, wie ein konkretes physikalisches Experiment durch eine Anwendung mit der HoloLens unterstützt werden kann. Dafür wurde ein Versuch aus der Elektrodynamik ausgewählt, Probleme identifiziert und eine Lösung erarbeitet. Diese soll nun im Hinblick auf die in der Problemstellung festgehaltenen Anforderungen bewertet werden. Die Ergebnisse werden dann im breiteren Kontext der Fragestellung diskutiert.\\


\begin{comment}

\begin{itemize}
\item Welche physikalischen Zusammenhänge werden vermittelt?
\item Wie 
\end{itemize}

Konkrete untergeordnete Frage:
\begin{center}
\textit{\textbf{Wie kann die HoloLens in einem konkreten Anwendungsfall eines physikalischen Versuches genutzt werden?}}
\end{center}
\subsection{Aufgabenstellung}

Im Rahmen der Arbeit soll anhand der HoloLens untersucht werden, wie diese in der Physik-Lehre eingesetzt werden kann, um physikalische Inhalte zu vermitteln. Insbesondere soll betrachtet werden, wie physikalische Experimente mittels Mixed Reality Anwendungen durch zusätzliche Inhalte angereichert werden können.\\

\par
Dazu sind zunächst die technischen Möglichkeiten und Voraussetzungen der HoloLens zu betrachten und in Zusammenhang mit dem Anwendungsfall zu bringen. Weiterhin sind bestehende Ansätze im Einsatz von Mixed Reality Technologie in der Lehre, besonders in der Physik-Lehre, herauszuarbeiten und einzuordnen.

Davon ausgehend soll der Fragestellung anhand eines konkreten Beispiels nachgegangen werden. Für einen ausgewählten Versuchsaufbau sind die darzustellenden Objekte und Informationen sowie das Zusammenspiel dieser mit dem aufgebauten Experiment, der Umgebung und den Nutzern zu entwickeln. Für den ausgewählten Anwendungsfall soll eine Umsetzung mit der HoloLens konzipiert, designet und prototypisch implementiert werden.
\end{comment}

\subsection{Ergebnisse}
Die Arbeit stellt einen Lösungsansatz vor, der 

\subsubsection{Unterstützung des Experimentes}
Die Lösung erweitert das gegebene Experiment um zusätzliche Inhalte. Die
\begin{itemize}
	\item Erweiterung des Versuches um Magnetfeld, Stromfluss, Kompass und weitere Informationen
	\item Physikalische Eigenschaften von Magnetfeld abgebildet
	\item Physikalische Zusammenhänge zwischen Stromfluss, Magnetfeld und Kompass umgesetzt
	\item Anwendung zeigt nicht sichtbare, reale physikalische Zusammenhänge des Experimentes
\end{itemize}

\subsubsection{Technische Umsetzung}
Im Hinblick auf die technischen Anforderungen wurde eine Liste von Qualitätskriterien zugrunde gelegt. Die Bewertung der Lösung anhand dieses Maßstabes ist Tabelle \ref{tab:tech_results} zu entnehmen. Der Einschätzung liegen die zu jedem Kriterium genannten Hinweise zur Bewertung zu Grunde.\\

\textbf{Qualitätskriterien}
\begin{landscape}
	\bgroup
	\setlength\extrarowheight{-2pt}
	\def\arraystretch{1.8}
	\begin{table}
		\centering
		\begin{tabular}{m{2.3cm}|m{15.5cm}|m{2cm}}
			Kriterium & Ergebnis & Bewertung\\
			\hline
			\hline
			Framerate & Durchgehend 60 FPS, keine Framedrops. Einzig bei Start/Stopp von Vuforia hängt die Anwendung kurz (<1s). & Optimal\\
			\hline
			Stabilität der Hologramme & Hologramme erscheinen durchgehend sehr stabil, Elemente liegen im Abstand von max. 20cm zu einem Spatial Anchor und der Stabilization Plane. Seltene, minimale Sprünge sowie Wippen der simulierten Feldlinien treten auf. & Fast optimal\\
			\hline
			Positionierung & Sehr genaue Positionierung über optischen Marker. Hologramme sind glaubhaft in die Spule eingebettet. Überschneidungen von wenigen Millimetern sind jedoch möglich. & Fast optimal\\
			\hline
			Komfortzone & Elemente liegen in Komfortzone (Winkel und Distanz), sofern eine geeignete Unterlage vorhanden ist. Design begünstigt den gewünschten Abstand. Minimale Distanz wird über Fading sichergestellt. & Optimal\\
			\hline
			Fokuswechsel & Kaum Neufokussierungen notwendig. Nur bei Ablesen des Kompass und der Stromrichtung. & Fast optimal\\
			\hline
			FOV-Grenzen & Darstellungen passen bei empfohlener Distanz vollständig in FOV. Durch Verankerung am realen Objekt verliert Nutzer den Kontext nicht. & Optimal\\
			\hline
			Anpassung an Nutzerposition & Text, Labels, Linien und Menüs richten sich zur Nutzerposition aus. Frei bewegliche Elemente (Menü, Progress Indikator) folgen Nutzer. & Optimal\\
			\hline
			Input Interaction Clarity & Vorhandene, einfache Standard-Interaktionsmechanismen genutzt und geringfügig erweitert. Konsistentes, akustisches Feedback. Jedoch keine Erklärung der möglichen Aktionen durch die Anwendung selbst. & Mittel\\
			\hline
			Interaktive Objekte & Vorhandene Buttons, Felder und Cursor genutzt. & Optimal\\
			\hline
			Ladevorgänge & Vorhandenen, animierten Progress Indikator verwendet. Lesbarkeit des Textes wird abgesichert. Vorgänge sind kurz, keine Angabe einer erwarteten Dauer notwendig. & Optimal\\
		\end{tabular}\caption{\label{tab:tech_results} Bewertung der Umsetzung anhand von Qualitätskriterien.}
	\end{table}
	\egroup
\end{landscape}

Insgesamt erfüllt die Applikation die Anforderungen fast vollständig bis vollständig. Einige Probleme konnten bereits im Design abgemildert oder umgangen werden. Das gilt z.B. für die Aspekte Komfortzone und FOV-Grenzen. Durch die platzsparende Anordnung der Elemente werden die Hologramme seltener abgeschnitten. Außerdem sind häufige Kopfbewegungen und Hinweise auf Elemente außerhalb des FOV so nicht notwendig. Und die Berücksichtigung einer geeigneten Distanz von Beginn an vermeidet bzw. verringert Probleme mit zu dicht positionierten Objekten.\\

Die Umsetzung unterstützt bereits im Design verankerte Maßnahmen weiter. Hier wurde z.B. bei der Festsetzung von Anzahl und Größe von Objekten auf die Einschränkungen der HoloLens Rücksicht genommen. Viele Probleme werden auch durch die Nutzung von vorgefertigten Objekten und Verhaltensweisen vermieden. Hier sind z.B. der Progress Indikator, Standard-Button und Standard-Shader zu nennen. Im Weiteren soll auf einzelne Aspekte kurz näher eingegangen werden.\\

\textit{Stabilität}\\
Die Stabilität der Objekte beeinflusst die Nutzererfahrung wesentlich, da durch die Einbettung in die Spule selbst geringe Abweichungen negativ auffallen. Bei einer umsichtigen Nutzung treten diesbezüglich wenige bis keine Probleme auf. Selten sind kleinere Sprünge festzustellen. Einzig die Darstellung der Feldlinien wippt bei vertikalen Kopfbewegungen leicht um den Mittelpunkt. Das ist der Tatsache geschuldet, dass die Feldlinien steil auf der Stabilisations-Ebene stehen und sich von dieser mehr als einen Meter ausbreiten. Der Effekt ist jedoch gering und nur auffällig, wenn er provoziert wird.\\

Allerdings hängt die Stabilität nicht unwesentlich von den Randbedingungen ab. Die weiter unten genannten Umstände des Tests erleichtern das Tracking und die Stabilisation. Unter weniger geeigneten Bedingungen kann die Stabilität und damit auch die Positionierung der Hologramme beeinträchtigt werden. Insbesondere kann es bei unerfahrenen Nutzern durch ungünstige Aktionen wie z.B. ruckartige Kopfbewegungen zu negativen Auswirkungen auf die Stabilität kommen.\\

\textit{Positionierung}\\
\begin{wrapfigure}{L}{0.5\textwidth}
	\centering
	\includegraphics[width=0.48\textwidth]{images/todo.jpg}
	\caption{Überlagerung virtuelle und reale Spule}
	\label{img:model-overlay}
\end{wrapfigure}
Einen Eindruck von der Güte der Positionierung lässt sich anhand von Abb. \ref{img:model-overlay} gewinnen. Das normalerweise nur in den Tiefenpuffer gerenderte Mesh der virtuellen Spule wird hier durch einen Standard-Shader sichtbar dargestellt. Die reale Spule ist dadurch fast gar nicht zu sehen. Zwar sind Abweichungen zwischen den beiden Objekten aus unterschiedlichen Winkeln unterschiedlich stark sichtbar, das gewählte Foto gibt jedoch einen guten Anhaltspunkt für die tatsächliche Nutzererfahrung wieder.\\

\textit{Performance}\\
\begin{wrapfigure}{R}{0.5\textwidth}
	\centering
	\includegraphics[width=0.48\textwidth]{images/todo.jpg}
	\caption{Performance Monitor}
	\label{img:performance}
\end{wrapfigure}

Einen Eindruck vom Ressourcenverbauch der App vermittelt der Screenshot in Abb. \ref{img:performance}. Zu sehen ist die Auslastung der Brille über einen Zeitraum von 60 Sekunden. Währenddessen wurde mehrfach zwischen den verschiedenen Darstellungen hin- und hergewechselt und das Menü aufgerufen.\\

TODO Ergebnisse ermitteln
\begin{itemize}
	\item Framerate
	\item CPU
	\item GPU
	\item Stromverbrauch
\end{itemize}

\textit{Testbedingungen}\\
Die Bewertung wurde anhand von Tests unter für die Anwendung günstigen Bedingungen durchgeführt. Der Versuchsaufbau wurde so positioniert, dass die umliegenden Objekte gut für das Tracking der Brille geeignet sind. Außerdem wurde die HoloLens vorher öfter im Testraum verwendet und hatte daher bereits ein gutes Modell des Raums zur Orientierung. Weiterhin wurde die Brille auf den Tester kalibriert und ruckartige oder unnötige Kopfbewegungen vermieden. 

\subsubsection{Feedback}
Hier Feedback kurz zusammenfassen, Aussagen allgemein zusammenfassen sowie Zitate angeben
\begin{itemize}
	\item Anwendung wurde mehreren Personen aus unterschiedlichen Gruppen vorgestellt, nur vereinzelt vorige Erfahrungen mit HoloLens vorhanden
	\item Physik Professoren, wissenschaftliche Mitarbeiter, Lehrer, Studenten, außerdem interessierte Einzelpersonen mit unterschiedlichen Hintergründen, insgesamt ca. 15 Personen
	\item Feedback durchgehend positiv
	\item "Kann ich Fotos davon haben"
	\item Kleines Sichtfeld wurde mehrfach angemerkt
	\item Vor allem die direkte Interaktion über die Spannungsquelle und die Feldlinien aus den Simulationsdaten wurden positiv aufgenommen
\end{itemize}


\subsection{Diskussion}

\subsubsection{Vor- und Nachteile}
\subsubsection{Erweiterbarkeit}
\begin{itemize}
	\item Anwendung kann erweitert werden, z.B. um Ablenkung von Elektronen in der Spule
	\item 3D Darstellungen der HoloLens genutzt für Magnetfeld der Spulen, auch für andere Magnetfelder anwendbar
	\item Kräftevektoren ebenfalls übertragbar
	\item Einbettung von Simulation denkbar und sinnvoll, denn das erlaubt den Nutzern What-If Szenarien durchzuspielen, sehr wichtig zum Lernen
	\item Weitere Lehrinhalte denkbar: z.B. Entstehung der Einzelfelder, Überlagerung der Einzelfelder, Ablenkung eines Elektronenstrahls, 
	\item Applikation ist parametrisiert: Andere physikalische Parameter möglich
	\item Variation von Spulengröße möglich, aber Parameter für die Darstellung müssen angepasst werden. Bei einer anderen Spule müsste das virtuelle Modell geändert werden.
	\item Leistung hat noch Reserven für mehr Darstellungen
	\item Architektur erlaubt Einfügen neuer Szenen
\end{itemize}

\subsubsection{Übertragbarkeit}
\begin{itemize}
	\item Annotationen auch bei anderen Versuchen denkbar
	\item Unterstützung beim Aufbau eines Versuches möglich
\end{itemize}
	
	
	
	
	
	
	
	
	
	
	
	
	
	