\section{Ergebnisse und Diskussion}
\label{sec-6}
\fbox{
\parbox{\linewidth}{
	\textit{Ziel des Kapitels:}\\
	Ergebnisse der Umsetzung vorstellen und auf die Fragestellung anwenden.
}}
Bogen zur Fragestellung schließen

Konkrete untergeordnete Frage:
\begin{center}
	\textit{\textbf{Wie kann die HoloLens in einem konkreten Anwendungsfall eines physikalischen Versuches genutzt werden?}}
\end{center}

\begin{comment}
\subsection{Aufgabenstellung}

Im Rahmen der Arbeit soll anhand der HoloLens untersucht werden, wie diese in der Physik-Lehre eingesetzt werden kann, um physikalische Inhalte zu vermitteln. Insbesondere soll betrachtet werden, wie physikalische Experimente mittels Mixed Reality Anwendungen durch zusätzliche Inhalte angereichert werden können.\\

\par
Dazu sind zunächst die technischen Möglichkeiten und Voraussetzungen der HoloLens zu betrachten und in Zusammenhang mit dem Anwendungsfall zu bringen. Weiterhin sind bestehende Ansätze im Einsatz von Mixed Reality Technologie in der Lehre, besonders in der Physik-Lehre, herauszuarbeiten und einzuordnen.

Davon ausgehend soll der Fragestellung anhand eines konkreten Beispiels nachgegangen werden. Für einen ausgewählten Versuchsaufbau sind die darzustellenden Objekte und Informationen sowie das Zusammenspiel dieser mit dem aufgebauten Experiment, der Umgebung und den Nutzern zu entwickeln. Für den ausgewählten Anwendungsfall soll eine Umsetzung mit der HoloLens konzipiert, designet und prototypisch implementiert werden.
\end{comment}

\subsection{Ergebnisse}
Design

\subsubsection{Unterstützung des Experimentes}
1. Lösung ganz kurz zusammenfassen\\
\begin{itemize}
	\item Erweiterung des Versuches um Magnetfeld, Stromfluss, Kompass und weitere Informationen
	\item Physikalische Eigenschaften von Magnetfeld abgebildet
	\item Physikalische Zusammenhänge zwischen Stromfluss, Magnetfeld und Kompass umgesetzt
	\item Anwendung zeigt nicht sichtbare, reale physikalische Zusammenhänge des Experimentes
	\item Nutzer kann frei mit dem Versuchsaufbau interagieren
	\item Einführung in Nutzung der HoloLens nicht notwendig
\end{itemize}

\subsubsection{Technische Umsetzung}
2. Lösung an technischen Kriterien messen\\
\textbf{Qualitätskriterien}
\begin{itemize}
	\item Framerate
	\item Stabilität
	\item Positionierung
	\item Betrachtungs Komfort Zone
	\item Tiefen Wechsel
	\item FOV Grenezn
	\item Anpassung an Nutzerposition
	\item Input Interaction Clarity
	\item Interaktive Objekte
	\item Ladevorgänge
\end{itemize}

\textbf{Performance}
Kurzen Überblick über Performance geben
\begin{itemize}
	\item Framerate
	\item CPU
	\item GPU
	\item Stromverbrauch
\end{itemize}


\subsubsection{Feedback}
Hier Feedback kurz zusammenfassen, Aussagen allgemein zusammenfassen sowie Zitate angeben
\begin{itemize}
	\item Anwendung wurde mehreren Personen aus unterschiedlichen Gruppen vorgestellt, nur vereinzelt vorige Erfahrungen mit HoloLens vorhandens
	\item Physik Professoren, wissenschaftliche Mitarbeiter, Lehrer, Studenten, außerdem interessierte Einzelpersonen mit unterschiedlichen Hintergründen, insgesamt ca. 15 Personen
	\item Feedback durchgehend positiv
	\item "Kann ich Fotos davon haben"
	\item Kleines Sichtfeld wurde mehrfach angemerkt
	\item Vor allem die direkte Interaktion über die Spannungsquelle und die Feldlinien aus den Simulationsdaten wurden positiv aufgenommens
\end{itemize}


\subsection{Diskussion}

\subsubsection{Vor- und Nachteile}
\subsubsection{Erweiterbarkeit}
\subsubsection{Übertragbarkeit}
\begin{itemize}
	\item Weitere Lehrinhalte denkbar: z.B. Entstehung der Einzelfelder, Überlagerung der Einzelfelder, Ablenkung eines Elektronenstrahls, 
	\item Applikation ist parametrisiert: Andere physikalische Parameter möglich
	\item Variation von Spulengröße möglich, aber Parameter für die Darstellung müssen angepasst werden. Bei einer anderen Spule müsste das virtuelle Modell geändert werden.
	\item Leistung hat noch Reserven für mehr Darstellungen
	\item Architektur erlaubt Einfügen neuer Szenen
\end{itemize}
\begin{itemize}
	\item Anwendung kann erweitert werden, z.B. um Ablenkung von Elektronen in der Spule
	\item 3D Darstellungen der HoloLens genutzt für Magnetfeld der Spulen, auch für andere Magnetfelder anwendbar
	\item Kräftevektoren ebenfalls übertragbar
	\item Einbettung von Simulation denkbar und sinnvoll, denn das erlaubt den Nutzern What-If Szenarien durchzuspielen, sehr wichtig zum Lernen
	\item ...
\end{itemize}
	
	
	
	
	
	
	
	
	
	
	
	
	
	
	