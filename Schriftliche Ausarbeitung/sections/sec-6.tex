\section{Ergebnisse und Diskussion}
\label{sec-6}
\fbox{
\parbox{\linewidth}{
	\textit{Ziel des Kapitels:}\\
	Ergebnisse der Umsetzung vorstellen und auf die Fragestellung anwenden.
}}	

\subsection{Ergebnisse}

\subsubsection{Unterstützung des Experimentes}
\begin{itemize}
	\item Erweiterung des Versuches um Magnetfeld, Stromfluss, Kompass und Informationen
	\item Physikalische Eigenschaften von Magnetfeld abgebildet
	\item Physikalische Zusammenhänge zwischen Stromfluss, Magnetfeld und Kompass umgesetzt
	\item Anwendung zeigt nicht sichtbare, reale physikalische Zusammenhänge des Experimentes
	\item Nutzer kann frei mit dem Versuchsaufbau interagieren
	\item Einführung in Nutzung der HoloLens nicht notwendig
\end{itemize}

\subsubsection{Technische Umsetzung}
\textbf{Qualitätskriterien}
\begin{itemize}
	\item Framerate
	\item Stabilität
	\item Positionierung
	\item Betrachtungs Komfort Zone
	\item Tiefen Wechsel
	\item FOV Grenezn
	\item Anpassung an Nutzerposition
	\item Input Interaction Clarity
	\item Interaktive Objekte
	\item Ladevorgänge
\end{itemize}

\textbf{Performance}
\begin{itemize}
	\item Framerate
	\item CPU
	\item GPU
	\item Stromverbrauch
\end{itemize}


\subsubsection{Feedback}
\begin{itemize}
	\item Qualitätskriterien
	\item Performance
	\item Qualitative Ergebnisse
	\item Feedback
\end{itemize}


\subsection{Diskussion}
\textbf{Erweiterbarkeit}
\begin{itemize}
	\item Weitere Lehrinhalte denkbar: z.B. Entstehung der Einzelfelder, Überlagerung der Einzelfelder, Ablenkung eines Elektronenstrahls, 
	\item Applikation ist parametrisiert: Andere physikalische Parameter möglich
	\item Variation von Spulengröße möglich, aber Parameter für die Darstellung müssen angepasst werden. Bei einer anderen Spule müsste das virtuelle Modell geändert werden.
	\item Leistung hat noch Reserven für mehr Darstellungen
	\item Architektur erlaubt Einfügen neuer Szenen
\end{itemize}
\begin{itemize}
	\item Anwendung kann erweitert werden, z.B. um Ablenkung von Elektronen in der Spule
	\item 3D Darstellungen der HoloLens genutzt für Magnetfeld der Spulen, auch für andere Magnetfelder anwendbar
	\item Kräftevektoren ebenfalls übertragbar
	\item Einbettung von Simulation denkbar und sinnvoll, denn das erlaubt den Nutzern What-If Szenarien durchzuspielen, sehr wichtig zum Lernen
	\item ...
\end{itemize}
	
	
	
	
	
	
	
	
	
	
	
	
	
	
	