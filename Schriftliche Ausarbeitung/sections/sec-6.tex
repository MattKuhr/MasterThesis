\section{Ergebnisse und Diskussion}
\label{sec-6}
\fbox{
\parbox{\linewidth}{
	\textit{Ziel des Kapitels:}\\
	Ergebnisse der Umsetzung vorstellen und auf die Fragestellung anwenden.
}}	

\subsection{Ergebnisse}

\begin{itemize}
	\item Qualitätskriterien
	\item Performance
	\item Qualitative Ergebnisse
	\item Feedback
\end{itemize}

\begin{itemize}
	\item Erweiterung des Versuches um Magnetfeld, Stromfluss, Kräftevektoren
	\item Nutzer hat die Hände frei und kann Spulen bewegen und vor allem drehen
	\item Anwendung mit intrinsic contextuality, da Darstellungen eingebettet sind
	\item Anwendung zeigt nicht sichtbare, reale physikalische Zusammenhänge des Experimentes
	\item Depth Cues zur besseren Tiefenwahrnehmungen genutzt
	\item Empfohlene Constraints berücksichtigt:
	\subitem Distanz, Größe, Geschwindigkeit, Blickwinkel
	\subitem Framerate
	\item Nicht umgesetzt werden konnte eine Einführung in die Gestensteuerung
	\item ...
\end{itemize}

\subsection{Diskussion}
\textbf{Erweiterbarkeit}
\begin{itemize}
	\item Weitere Lehrinhalte denkbar: z.B. Entstehung der Einzelfelder, Überlagerung der Einzelfelder, Ablenkung eines Elektronenstrahls, 
	\item Applikation ist parametrisiert: Andere physikalische Parameter möglich
	\item Variation von Spulengröße möglich, aber Parameter für die Darstellung müssen angepasst werden. Bei einer anderen Spule müsste das virtuelle Modell geändert werden.
	\item Leistung hat noch Reserven für mehr Darstellungen
	\item Architektur erlaubt Einfügen neuer Szenen
\end{itemize}
\begin{itemize}
	\item Anwendung kann erweitert werden, z.B. um Ablenkung von Elektronen in der Spule
	\item 3D Darstellungen der HoloLens genutzt für Magnetfeld der Spulen, auch für andere Magnetfelder anwendbar
	\item Kräftevektoren ebenfalls übertragbar
	\item Einbettung von Simulation denkbar und sinnvoll, denn das erlaubt den Nutzern What-If Szenarien durchzuspielen, sehr wichtig zum Lernen
	\item ...
\end{itemize}
	
	
	
	
	
	
	
	
	
	
	
	
	
	
	