\section{Design}
\label{sec-4}
\fbox{
	\parbox{\linewidth}{
		\textit{Ziel des Kapitels:}\\
		Eigene Lösungsidee vorstellen, Prinzip erläutern.\\[6px]
}}\\

\subsection{Grundidee}
\textbf{Mixed Reality: Einbettung und Zusammenspiel von realen und virtuellen Objekten}\\
Grundidee, motiviert aus bestehenden Arbeiten\\

\begin{itemize}[topsep=-2px]
	\setlength{\itemsep}{-5pt}
	\item Integration von realen und virtuellen Objekten in eine Mixed Reality Anwendung
	\item Virtuelle Objekte werden in den Versuchsaufbau eingebettet und interagieren mit diesem in Echtzeit
	\item Gemessene, berechnete und beobachtete physikalische Vorgänge werden somit in einer Anwendung integriert
	\item Nutzer kann sich mit der HoloLens frei im Raum bewegen und mit der Anwendung sowie den Gerätschaften interagieren
	\item Zentral ist dabei die Darstellung der entstehenden Magnetfelder
\end{itemize}

\subsection{Darstellungen}
\textbf{Magnetfeld}
\begin{itemize}[topsep=-2px]
	\setlength{\itemsep}{-5pt}
	\item Magnetfeld von Erde und Spule dreidimensional darstellen und räumlich in die Spule einbetten
	\item Einbettung durch Verdeckungsberechnung von realer Spule und virtueller Geometrie verbessern
	\item Sowohl die Darstellung über Feldlinien als auch die über Vektoren anbieten
	\item Für beide Varianten wird 3D-Geometrie genutzt
	\item Objekte verändern sich dynamisch anhand von Messwerten
	\item Darstellung beschränkt sich auf das Zentrum der Spule
	\item Eine davon unabhängige, eigene Darstellung von Simulationsdaten zeigt eine numerische Lösung für das gesamte Feld in einer ausgewählten Ebene
\end{itemize}
\vspace{8px}

\textbf{Stromfluss}
\begin{itemize}[topsep=-2px]
	\setlength{\itemsep}{-5pt}
	\item Direkt an den Spulen verankerte Indikatoren zeigen die Flussrichtung im Raum an
	\item Indikatoren nur sichtbar, wenn Strom fließt, zeigen so also auch an, ob die Spule unter Strom steht
	\item Kennzeichnung der Konnektoren mit Plus und Minus anhand von Labeln
\end{itemize}
\vspace{8px}

\textbf{Kompass}
\begin{itemize}[topsep=-2px]
	\setlength{\itemsep}{-5pt}
	\item Kompass entsteht aus Kombination von realer Magnetnadel und virtueller Skala
	\item Nadel wird mittig in der Spule aufgestellt
	\item Virtueller Ring mit Markierungen im 45° Abstand um die Spule herum dient als Skala	
	\item Virtuelle Nord-Süd sowie 45° Linie als Sehnen im Kreis durch das Zentrum der Spule
	\item Weitere Linie für theoretische Auslenkung, ebenfalls durchs Zentrum
	\item Die drei Diagonalen sind jeweils in der Mitte ausgeschnitten, damit der Kompass sichtbar bleibt
	\item Gradzahl der theoretischen Nadel wird am Kreis angezeigt
\end{itemize}
\vspace{8px}

\textbf{Weitere Informationen}
\begin{itemize}[topsep=-2px]
	\setlength{\itemsep}{-5pt}
	\item Numerische Werte (gemessen oder berechnet) werden auf einem Datenpanel dargestellt
	\item Physikalische Parameter werden als Labels an die entsprechenden Elemente angehängt
\end{itemize}
\vspace{4px}

\subsection{Interaktion}
\textbf{Hauptteil der Anwendung - Durchführung des Versuches}\\
Die Interaktion ist einfach gehalten. Während der Durchführung des Versuches sind vor allem zwei Aktionen möglich:
\begin{enumerate}[topsep=-2px]
	\setlength{\itemsep}{-5pt}
	\item Veränderung des Stromflusses über den Regler der Spannungsquelle
	\item Wechsel zwischen den beiden Darstellungsmodellen
\end{enumerate}
\vspace{4px}

Die Anwendung passt sich automatisch an den sich ändernden Stromfluss an. Außerdem reagiert die Applikation auf Bewegungen des Nutzers und passt ggf. Darstellungen entsprechend der Position und Blickrichtung des Nutzers an.\\[4px]

\textbf{Drumherum der Anwendung - Start/Stopp/Restart und Setup}
\begin{itemize}[topsep=-2px]
	\setlength{\itemsep}{-5pt}
	\item Menu mit Start der Hauptanwendung und Optionen
	\item Über Optionen können wichtige Parameter des Versuches angepasst werden
	\item Das betrifft in erster Linie die IP-Adresse des Servers sowie angenommene Werte für die Flussdichte des Erdmagnetfeldes
	\item Optionen werden persistent auf der HoloLens gespeichert und lassen sich ggf. auf Default-Werte zurücksetzen
	\item Menu lässt sich jederzeit wieder aufrufen
\end{itemize}

\subsection{Zusammenhang: Problem \& Design}
Beschreiben: 
\begin{itemize}
	\item Wie nutzt das Konzept die Möglichkeiten der HoloLens und wie navigiert es um technische Einschränkungen herum?
	\item Wie bedient das Design die Anforderungen?
\end{itemize}

\textbf{Positionierung und Verdeckung}
Löst:
\begin{itemize}
	\item Ermöglicht die gewünschte Einbettung
\end{itemize}

\textbf{Echtzeitdaten}
Löst:
\begin{itemize}
	\item Interaktivität und Erforschbarkeit
	\item Einbettung
	\item Ermöglicht überlagerung von theoretischen Ergebnissen mit realen
\end{itemize}

\textbf{Zwei MFeld Darstellungen}
Löst: 
\begin{itemize}
	\item Nutzt Vor- und Nachteile der beiden Varianten
	\item Anpassbarkeit
	\item Vergleichbarkeit Theoretische vs. Experimentelle Ergebnisse
\end{itemize}

\textbf{Statusinformationen}
Löst: 
\begin{itemize}
	\item Zentralisiert wichtige Infos und ermöglicht Bewegungsfreiheit
	\item Direkte Vergleichbarkeit theoretischer Ergebnisse mit tatsächlichen Ergebnissen
\end{itemize}

\textbf{Kompass}
Löst: 
\begin{itemize}
	\item Nordrichtung ist direkt ablesbar
	\item Theoretische Auslenken direkt ablesbar
	\item Durch Markierungen auch 45° direkt ablesbar
	\item Differenz zwischen theoretischer Linie und realem Kompass sehr gut zu sehen
	\item Durch Ring außen weniger Clutter (Quelle unbekannt)
\end{itemize}

\textbf{Stromfluss-Indikatoren}
Löst: 
\begin{itemize}
	\item Stromrichtung ist aus allen Perspektiven ablesbar
	\item Richtung ist 3dimensional eingebettet
	\item Indikatoren vermeiden Clutter, da sie auf der Spule aufliegen, und so nur Teile des Sichtfeldes einnehmenm, die von der Spule ohnehin verdeckt sind
	\item Durch Ein/Ausblenden bei Perspektivwechsel wird die Information immer nur an dem gerade relevanten Punkt angezeigt
\end{itemize}

\textbf{Plus-Minus-Labels}
Löst: 
\begin{itemize}
	\item Zeigt Plus- und Minuspole an und ergänzt so die Fluss-Indikatoren um wichtige Information
	\item Tooltip-Style ermöglicht Lesbarkeit aus verschiedenen Positionen
	\item Verankerung unten, außerhalb der Spule vermeidet, dass die Labels die Darstellungen des MFeldes behindern
	\item Farbe passend zur realen Farbe gewählt für Kontinuität und Zusammenhang
	\item Ständige Sichtbarkeit der Information nicht unbedingt notwendig, daher Verdeckung durch Spule ok, das unterstützt noch die Einbettung
	\item Vorhandenen ToolTip-Style verwendet für Consistency
\end{itemize}

\textbf{Sprachkommandos}
Löst: 
\begin{itemize}
	\item Nutzer muss nicht zunächst Handgesten lernen
	\item Nur wenige Kommandos notwendig
	\item Kein Cursor notwendig
	\item Einfach auszusprechende Begriffe sind auch für Nutzer geeignet, die nicht so gut im Englischen zu Hause sind
\end{itemize}

\subsection{eher technische Richtung}

\textbf{Near Plane Fading}
Löst: 
\begin{itemize}
	\item Minnimum Distanz
	\item Behinderung bei Interaktion mit Versuchsaufbau
	\item Ermöglicht ''Eintauchen'', reduziert Clutter
	\item Bessere UX als Clipping
\end{itemize}

\textbf{Minimum Value Fading}
Löst: 
\begin{itemize}
	\item Darstellungen sind erst ab minimal Werten sinnvoll
	\item Übergang zwischen keiner Darstellung und der Darstellung von minimal Werten über Transparenz
	\item Dadurch erkennbarer, fließender Übergang
	\item Konsistent für alle betroffenen Darstellungen
	\item Transparenz wird nicht für die Abbildung von Informationen verwendet, damit es hier keine Konflikte gibt
	\item Bessere UX als hard cut
\end{itemize}

\textbf{Kein sichtbarer Cursor während des Hauptteils der Anwendung}
\begin{itemize}
	\item Kaum Objekte auszuwählen
	\item Klicks werden akustisch bestätigt
\end{itemize}

\textbf{Erkennung der Position über eigene Szene in Zusammenarbeit mit dem Nutzer}
\begin{itemize}
	\item Nutzer versteht die Funktionsweise der Anwendung
	\item Nutzer erhält Feedback über den Prozess der Positionsbestimmung von der Anwendung
	\item Nutzer kann sein Verhalten anpassen und so den Prozess unterstützen
	\item Bessere Kalibrierung möglich
	\item Tracking nur für kurze Zeit erforderlich, das spart viel Ressourcen
	\item Marker verdeckt kein Teil des Sichtfeldes, kann nach Positionierung auch entfernt werden
\end{itemize}

\textbf{Künstliche Beleuchtung}
\begin{itemize}
	\item Virtuelle Beleuchtung von Oben bildet echte Lichtverhältnisse grob nach
	\item Unterstützt die Einbettung der 3D Objekte im Raum
\end{itemize}





