\section{Konzept Prinzipieller Lösungsansatz}
\label{sec-4}
\fbox{
	\parbox{\linewidth}{
		\textit{Ziel des Kapitels:}\\
		Eigene Lösungsidee vorstellen, Prinzip erläutern.\\[6px]
		\textit{Inhalte:}
		\begin{itemize}
			\item Lösungsidee
		\end{itemize}
		
		\textit{(Optional) Literatur:}	
		%\begin{itemize}
		%\end{itemize}
}}\\

Idee:
\begin{itemize}
	\item Position der Spulen mit optischem Marker bestimmen
	\item Homogenen Teil des Magnetfeldes durch Pfeile repräsentieren
	\item Pfeile räumlich in die Spulen einbetten
	\item Stromfluss durch sich bewegende 2D Elektronen-Icons auf Spulen darstellen
	\item Mehrere Modi mit jeweils unterschiedlichen dargestellten Objekten
	\item Menu per Click-and-Hold Geste aufrufbar
	\item Optional auch Sprachsteuerung möglich
	\item Kurze, optionale Einführung in Gesten
\end{itemize}

