\section{Design}
\label{sec-4}
\fbox{
	\parbox{\linewidth}{
		\textit{Ziel des Kapitels:}\\
		Eigene Lösungsidee vorstellen, Prinzip erläutern.\\[6px]
}}\\

\subsection{Grundidee}
\textbf{Mixed Reality: Einbettung und Zusammenspiel von realen und virtuellen Objekten}
Grundidee, motiviert aus bestehenden Arbeiten\\

\begin{itemize}
	\item Magnetfeld als räumlich in die Spule eingebettete, dreidimensionale Geometrie anzeigen
	\item Einbettung durch Verdeckungsberechnung von realer Spule und virtueller Geometrie verbessern
	\item Zwei verschiedene schematische Darstellungen für Magnetfeld anbieten: Vektor-Modell und Feldlinien-Modell
	\item Sensordaten in Echtzeit an HoloLens übermitteln und Geometrie ad-hoc berechnen
	\item Statusinformationen als Text-UI Elemente anzeigen
	\item Stromfluss durch 2D Geometrie, verankert an den beiden Spulen-Teilen, visualisieren
	\item Eingebetteter 3D-Kompass mit Markierungen anzeigen
\end{itemize}

\subsection{Darstellungen}

\subsection{Interaktion}

\subsection{Zusammenhang: Problem \& Design}
Beschreiben: 
\begin{itemize}
	\item Wie nutzt das Konzept die Möglichkeiten der HoloLens und wie navigiert es um technische Einschränkungen herum?
	\item Wie bedient das Design die Anforderungen?
\end{itemize}

\textbf{Positionierung und Verdeckung}
Löst:
\begin{itemize}
	\item Ermöglicht die gewünschte Einbettung
\end{itemize}

\textbf{Echtzeitdaten}
Löst:
\begin{itemize}
	\item Interaktivität und Erforschbarkeit
	\item Einbettung
	\item Ermöglicht überlagerung von theoretischen Ergebnissen mit realen
\end{itemize}

\textbf{Zwei MFeld Darstellungen}
Löst: 
\begin{itemize}
	\item Nutzt Vor- und Nachteile der beiden Varianten
	\item Anpassbarkeit
	\item Vergleichbarkeit Theoretische vs. Experimentelle Ergebnisse
\end{itemize}

\textbf{Statusinformationen}
Löst: 
\begin{itemize}
	\item Zentralisiert wichtige Infos und ermöglicht Bewegungsfreiheit
	\item Direkte Vergleichbarkeit theoretischer Ergebnisse mit tatsächlichen Ergebnissen
\end{itemize}

\textbf{Kompass}
Löst: 
\begin{itemize}
	\item Nordrichtung ist direkt ablesbar
	\item Theoretische Auslenken direkt ablesbar
	\item Durch Markierungen auch 45° direkt ablesbar
	\item Differenz zwischen theoretischer Linie und realem Kompass sehr gut zu sehen
	\item Durch Ring außen weniger Clutter (Quelle unbekannt)
\end{itemize}

\textbf{Stromfluss-Indikatoren}
Löst: 
\begin{itemize}
	\item Stromrichtung ist aus allen Perspektiven ablesbar
	\item Richtung ist 3dimensional eingebettet
	\item Indikatoren vermeiden Clutter, da sie auf der Spule aufliegen, und so nur Teile des Sichtfeldes einnehmenm, die von der Spule ohnehin verdeckt sind
	\item Durch Ein/Ausblenden bei Perspektivwechsel wird die Information immer nur an dem gerade relevanten Punkt angezeigt
\end{itemize}

\textbf{Plus-Minus-Labels}
Löst: 
\begin{itemize}
	\item Zeigt Plus- und Minuspole an und ergänzt so die Fluss-Indikatoren um wichtige Information
	\item Tooltip-Style ermöglicht Lesbarkeit aus verschiedenen Positionen
	\item Verankerung unten, außerhalb der Spule vermeidet, dass die Labels die Darstellungen des MFeldes behindern
	\item Farbe passend zur realen Farbe gewählt für Kontinuität und Zusammenhang
	\item Ständige Sichtbarkeit der Information nicht unbedingt notwendig, daher Verdeckung durch Spule ok, das unterstützt noch die Einbettung
	\item Vorhandenen ToolTip-Style verwendet für Consistency
\end{itemize}

\textbf{Sprachkommandos}
Löst: 
\begin{itemize}
	\item Nutzer muss nicht zunächst Handgesten lernen
	\item Nur wenige Kommandos notwendig
	\item Kein Cursor notwendig
	\item Einfach auszusprechende Begriffe sind auch für Nutzer geeignet, die nicht so gut im Englischen zu Hause sind
\end{itemize}

\subsection{eher technische Richtung}

\textbf{Near Plane Fading}
Löst: 
\begin{itemize}
	\item Minnimum Distanz
	\item Behinderung bei Interaktion mit Versuchsaufbau
	\item Ermöglicht ''Eintauchen'', reduziert Clutter
	\item Bessere UX als Clipping
\end{itemize}

\textbf{Minimum Value Fading}
Löst: 
\begin{itemize}
	\item Darstellungen sind erst ab minimal Werten sinnvoll
	\item Übergang zwischen keiner Darstellung und der Darstellung von minimal Werten über Transparenz
	\item Dadurch erkennbarer, fließender Übergang
	\item Konsistent für alle betroffenen Darstellungen
	\item Transparenz wird nicht für die Abbildung von Informationen verwendet, damit es hier keine Konflikte gibt
	\item Bessere UX als hard cut
\end{itemize}

\textbf{Kein sichtbarer Cursor während des Hauptteils der Anwendung}
\begin{itemize}
	\item Kaum Objekte auszuwählen
	\item Klicks werden akustisch bestätigt
\end{itemize}

\textbf{Erkennung der Position über eigene Szene in Zusammenarbeit mit dem Nutzer}
\begin{itemize}
	\item Nutzer versteht die Funktionsweise der Anwendung
	\item Nutzer erhält Feedback über den Prozess der Positionsbestimmung von der Anwendung
	\item Nutzer kann sein Verhalten anpassen und so den Prozess unterstützen
	\item Bessere Kalibrierung möglich
	\item Tracking nur für kurze Zeit erforderlich, das spart viel Ressourcen
	\item Marker verdeckt kein Teil des Sichtfeldes, kann nach Positionierung auch entfernt werden
\end{itemize}

\textbf{Künstliche Beleuchtung}
\begin{itemize}
	\item Virtuelle Beleuchtung von Oben bildet echte Lichtverhältnisse grob nach
	\item Unterstützt die Einbettung der 3D Objekte im Raum
\end{itemize}





