\section{Design}
\label{sec-4}
\fbox{
	\parbox{\linewidth}{
		\textit{Ziel des Kapitels:}\\
		Eigene Lösungsidee vorstellen, Prinzip erläutern.\\[6px]
		\textit{Inhalte:}
		\begin{itemize}
			\item Lösungsidee
		\end{itemize}
		
		\textit{(Optional) Literatur:}	
		%\begin{itemize}
		%\end{itemize}
}}\\

Idee:
\begin{itemize}
	\item Magnetfeld als räumlich in die Spule eingebettete, dreidimensionale Geometrie anzeigen
	\item Einbettung durch Verdeckungsberechnung von realer Spule und virtueller Geometrie verbessern
	\item Zwei verschiedene schematische Darstellungen für Magnetfeld anbieten: Vektor-Modell und Feldlinien-Modell
	\item Sensordaten in Echtzeit an HoloLens übermitteln und Geometrie ad-hoc berechnen
	\item Statusinformationen als Text-UI Elemente anzeigen
	\item Stromfluss durch 2D Geometrie, verankert an den beiden Spulen-Teilen, visualisieren
	\item Eingebetteter 3D-Kompass mit Markierungen anzeigen
\end{itemize}

Beschreiben: 
\begin{itemize}
	\item Wie nutzt das Konzept die Möglichkeiten der HoloLens und wie navigiert es um technische Einschränkungen herum?
	\item Wie bedient das Design die Anforderungen?
\end{itemize}

