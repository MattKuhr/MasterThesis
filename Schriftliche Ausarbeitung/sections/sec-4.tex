\section{Design}
\label{sec-4}
\fbox{
	\parbox{\linewidth}{
		\textit{Ziel des Kapitels:}\\
		Eigene Lösungsidee vorstellen, Prinzip erläutern.\\[6px]
}}\\

\subsection{Grundidee}
\textbf{Mixed Reality: Einbettung und Zusammenspiel von realen und virtuellen Objekten}
Grundidee, motiviert aus bestehenden Arbeiten\\

\begin{itemize}
	\item Magnetfeld als räumlich in die Spule eingebettete, dreidimensionale Geometrie anzeigen
	\item Einbettung durch Verdeckungsberechnung von realer Spule und virtueller Geometrie verbessern
	\item Zwei verschiedene schematische Darstellungen für Magnetfeld anbieten: Vektor-Modell und Feldlinien-Modell
	\item Sensordaten in Echtzeit an HoloLens übermitteln und Geometrie ad-hoc berechnen
	\item Statusinformationen als Text-UI Elemente anzeigen
	\item Stromfluss durch 2D Geometrie, verankert an den beiden Spulen-Teilen, visualisieren
	\item Eingebetteter 3D-Kompass mit Markierungen anzeigen
\end{itemize}

\subsection{Darstellungen}

\subsection{Interaktion}

\subsection{Zusammenhang: Problem \& Design}
Beschreiben: 
\begin{itemize}
	\item Wie nutzt das Konzept die Möglichkeiten der HoloLens und wie navigiert es um technische Einschränkungen herum?
	\item Wie bedient das Design die Anforderungen?
\end{itemize}

\textbf{Positionierung und Verdeckung}
Löst:
\begin{itemize}
	\item Ermöglicht die gewünschte Einbettung
\end{itemize}

\textbf{Echtzeitdaten}
Löst:
\begin{itemize}
	\item Interaktivität und Erforschbarkeit
	\item Einbettung
\end{itemize}

\textbf{Zwei MFeld Darstellungen}
Löst: 
\begin{itemize}
	\item Nutzt Vor- und Nachteile der beiden Varianten
	\item Anpassbarkeit
\end{itemize}

\textbf{Near Plane Fading}
Löst: 
\begin{itemize}
	\item Minnimum Distanz
	\item Behinderung bei Interaktion mit Versuchsaufbau
	\item Ermöglicht ''Eintauchen'', reduziert Clutter
	\item Bessere UX als Clipping
\end{itemize}

\textbf{Statusinformationen}
Löst: 
\begin{itemize}
	\item Zentralisiert wichtige Infos und ermöglicht Bewegungsfreiheit
	\item Direkte Vergleichbarkeit theoretischer Ergebnisse mit tatsächlichen Ergebnissen
\end{itemize}

\textbf{Kompass}
Löst: 
\begin{itemize}
	\item Cognitive Load
\end{itemize}




