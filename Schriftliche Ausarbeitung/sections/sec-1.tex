\section{Einleitung}
\label{sec-1}
\begin{comment}
\begin{center}
	\fbox{
		\parbox{0.9\linewidth}{
			\textit{Ziel des Kapitels:}\\
			Arbeit motivieren, an die Fragestellung heranführen und diese formulieren\\
			\textit{Inhalte:}
			\begin{itemize}
				\item MR kurz vorstellen, aktuelle Bedeutung hervorheben, HoloLens als aktuelles Device und Anhaltspunkt nutzen
				\item Anwendung in der Physik Lehre einleiten und motivieren
				\item Fragestellung, Ziel und Struktur der Arbeit
			\end{itemize}
			
			\textit{Wichtige Literatur:}	
			\begin{itemize}
				\item Physics holo.lab learning experience: using smartglasses for augmented reality labwork to foster the concepts of heat conduction \cite{Strzys18}
				\item Augmenting Microsoft's HoloLens with vuforia tracking for neuronavigation \cite{Frantz18}
				\item HoloMuse: Enhancing Engagement with Archaeological Artifacts Through Gesture-Based Interaction with Holograms \cite{Pollalis17}
			\end{itemize}
	}}
\end{center}
\end{comment}

TODO...

\subsection{Motivation}
\label{sec-1-1}
TODO...
\begin{itemize}
	\item Augmented Reality ein aktuelles Innovationsgebiet
	\item Aktuelle Bedeutung durch Fortschritte bei Hardware und KI
	\item HoloLens erweitert Realität durch virtuelle 3D Objekte
	\item Einsatz in der Physik interessant
	\item Existierende Beispiele: Thermodynamik + HoloLens, Elektromagnetismus und WebCam-AR
	\item Versuch Helmholtz-Spule aus 2. Beispiel aufgreifen
	\item Problem: Wie kann HoloLens genutzt werden, um Experiment durch virtuelle Darstellungen anzureichern?
	\item Motivation z.B. Einbinden von Informationen zu Versuchsaufbau, Visualisierung nicht sichtbarer physikalischer Größen und Vorgänge z.B. Magnetfeld und Stromfluss
\end{itemize}

\begin{comment}
Nicht zuletzt durch leistungsstärkere Hardware und Fortschritte im Bereich der künstlichen Intelligenz haben Augemented und Virtual Reality in den letzten Jahren an Bedeutung gewonnen.
\\
Physikalische Experimente durch virtuelle Darstellungen anzureichern und so besser und intuitiver verständlich zu machen, ist kein völlig neuer Ansatz. So stellen Strzys et. al. eine Anwendung mit der HoloLens im Bereich der Thermodynamik vor, bei der das gemessene Wärmeprofil eines erhitzten Metallstabes virtuell mit Hilfe der HoloLens auf den Stab gelegt wird \cite{Strzys17}. Und Buchau et. al. präsentieren eine Lösung, die unter anderem das Magnetfeld zweier Helmholtz-Spulen in das Echtzeitbild der Webcam zeichnet \cite{Buchau09}.\\ 
\end{comment}

\subsection{Fragestellung und prinzipieller Lösungsansatz}
\label{sec-1-2}
Übergeordnete wissenschaftliche Fragestellung:
\begin{center}
	\textit{\textbf{Kann die HoloLens in der Physik-Lehre eingesetzt werden, um physikalische Zusammenhänge zu vermitteln?}}
\end{center}

Konkrete untergeordnete Frage der Arbeit:
\begin{center}
	\textit{\textbf{Wie kann die HoloLens in dem konkreten Anwendungsfall eines physikalischen Versuches genutzt werden?}}
\end{center}

\begin{comment}
\subsection{Aufgabenstellung}

Im Rahmen der Arbeit soll anhand der HoloLens untersucht werden, wie diese in der Physik-Lehre eingesetzt werden kann, um physikalische Inhalte zu vermitteln. Insbesondere soll betrachtet werden, wie physikalische Experimente mittels Mixed Reality Anwendungen durch zusätzliche Inhalte angereichert werden können.\\

\par
Dazu sind zunächst die technischen Möglichkeiten und Voraussetzungen der HoloLens zu betrachten und in Zusammenhang mit dem Anwendungsfall zu bringen. Weiterhin sind bestehende Ansätze im Einsatz von Mixed Reality Technologie in der Lehre, besonders in der Physik-Lehre, herauszuarbeiten und einzuordnen.

Davon ausgehend soll der Fragestellung anhand eines konkreten Beispiels nachgegangen werden. Für einen ausgewählten Versuchsaufbau sind die darzustellenden Objekte und Informationen sowie das Zusammenspiel dieser mit dem aufgebauten Experiment, der Umgebung und den Nutzern zu entwickeln. Für den ausgewählten Anwendungsfall soll eine Umsetzung mit der HoloLens konzipiert, designet und prototypisch implementiert werden.
\end{comment}

\subsection{Aufbau der Arbeit}
\label{sec-1-3}
Die vorliegende Arbeit ist wie folgt aufgebaut. Kapitel \ref{sec-2} erläutert notwendige Hintergrundinformationen zur Technik der HoloLens, deren Anwendung in der Lehre sowie die physikalischen Hintergründe des Versuches. Kapitel \ref{sec-3} kristallisiert die Problemstellung und Anforderungen für den konkreten Anwendungsfall heraus. In Abschnitt \ref{sec-4} werden die gewählten Lösungsansätze vorgestellt, deren Umsetzung in Kapitel \ref{sec-5} erörtert wird. Kapitel \ref{sec-6} diskutiert diese dann im Rahmen der Fragestellung. Abschließend fasst Kapitel \ref{sec-7} die Ergebnisse zusammen und zieht ein Fazit.