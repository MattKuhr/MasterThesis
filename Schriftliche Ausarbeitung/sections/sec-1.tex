\section{Einleitung}
\label{sec-1}
Augmented Reality hat in den letzten Jahren immer mehr an Bedeutung gewonnen. Nicht zuletzt durch Fortschritte im Bereich der künstlichen Intelligenz und immer leistungsfähiger werdender Hardware eröffnen sich neue Möglichkeiten, um die Realität digital zu erweitern. 

 In vielen Bereichen existieren bereits Anwendungen, die auf Augmented Reality Technologie aufbauen. 

\textbf{HoloLens}\\
HoloLens erweitert Realität durch virtuelle 3D Objekte.

Augmenting Microsoft's HoloLens with vuforia tracking for neuronavigation \cite{Frantz18}
HoloMuse: Enhancing Engagement with Archaeological Artifacts Through Gesture-Based Interaction with Holograms \cite{Pollalis17}
Bienen \cite{Nguyen17}
HoloCPR: Designing and Evaluating a Mixed Reality Interface for Time-Critical Emergencies \cite{Johnson18}

\textbf{Anwendung Physik}\\
Einsatz in der Physik interessant
Existierende Beispiele: Thermodynamik + HoloLens, Elektromagnetismus und WebCam-AR 


Physikalische Experimente durch virtuelle Darstellungen anzureichern und so besser und intuitiver verständlich zu machen, ist kein völlig neuer Ansatz. So stellen Strzys et. al. eine Anwendung mit der HoloLens im Bereich der Thermodynamik vor, bei der das gemessene Wärmeprofil eines erhitzten Metallstabes virtuell mit Hilfe der HoloLens auf den Stab gelegt wird \cite{Strzys17}. Und Buchau et. al. präsentieren eine Lösung, die unter anderem das Magnetfeld zweier Helmholtz-Spulen in das Echtzeitbild der Webcam zeichnet \cite{Buchau09}.\\

\cite{Amiraslanov18}.  \cite{Javaheri18}

Motivation z.B. Einbinden von Informationen zu Versuchsaufbau, Visualisierung nicht sichtbarer physikalischer Größen und Vorgänge z.B. Magnetfeld und Stromfluss


\subsection{Fragestellung und prinzipieller Lösungsansatz}
\label{sec-1-2}
Die HoloLens ist ein AR-Device mit eigenen technischen Eigenschaften und Einschränkungen, die Auswirkungen darauf haben, ob und wie das Gerät in verschiedenen Anwendungsszenarien genutzt werden kann. Zum Einsatz im Bereich der Physik finden sich in der gängigen Literatur nur wenige Beispiele. Wie die Brille in der physikalischen Ausbildung dabei unterstützen kann, Konzepte und Zusammenhänge zu vermitteln, ist daher eine interessante Fragestellung. Die vorliegende Arbeit dringt in das Gebiet vor, indem sich näher mit der Unterstützung von Laborversuchen durch die HoloLens befasst wird. Dies geschieht anhand eines konkreten Beispiels: Die experimentelle Messung des Erdmagnetfeldes mit Hilfe einer Helmholtz-Spule.\\

Die konkrete Fragestellung dieser Arbeit lautet daher:
\begin{center}
	\textit{\textbf{Wie kann die HoloLens in dem konkreten Anwendungsfall eines physikalischen Versuches genutzt werden?}}
\end{center}

Es wird untersucht, mit welchen virtuellen Elementen der Versuch durch die HoloLens angereichert werden kann und wie diese Integration erfolgt. Der Lösungsansatz basiert darauf, die technischen Möglichkeiten der AR-Technologie zu nutzen, um nicht direkt sichtbare physikalische Eigenschaften in ihrem realen Kontext anzuzeigen und durch eine räumliche und zeitliche Einbettung in Zusammenhang zum Versuchsaufbau zu setzen. Gleichzeitig werden jedoch technische Limitierungen berücksichtigt, um negative Auswirkungen auf die Nutzererfahrung zu vermeiden. Ziel ist es, eine AR-Lösung mit der HoloLens zu entwickeln, die auf der einen Seite wichtige physikalische Eigenschaften und Zusammenhänge sichtbar und erfahrbar macht, und auf der anderen Seite eine komfortable Nutzung ermöglicht.

\begin{comment}
\subsection{Aufgabenstellung}

Im Rahmen der Arbeit soll anhand der HoloLens untersucht werden, wie diese in der Physik-Lehre eingesetzt werden kann, um physikalische Inhalte zu vermitteln. Insbesondere soll betrachtet werden, wie physikalische Experimente mittels Mixed Reality Anwendungen durch zusätzliche Inhalte angereichert werden können.\\

\par
Dazu sind zunächst die technischen Möglichkeiten und Voraussetzungen der HoloLens zu betrachten und in Zusammenhang mit dem Anwendungsfall zu bringen. Weiterhin sind bestehende Ansätze im Einsatz von Mixed Reality Technologie in der Lehre, besonders in der Physik-Lehre, herauszuarbeiten und einzuordnen.

Davon ausgehend soll der Fragestellung anhand eines konkreten Beispiels nachgegangen werden. Für einen ausgewählten Versuchsaufbau sind die darzustellenden Objekte und Informationen sowie das Zusammenspiel dieser mit dem aufgebauten Experiment, der Umgebung und den Nutzern zu entwickeln. Für den ausgewählten Anwendungsfall soll eine Umsetzung mit der HoloLens konzipiert, designet und prototypisch implementiert werden.
\end{comment}

\subsection{Aufbau der Arbeit}
\label{sec-1-3}
Die vorliegende Arbeit ist wie folgt aufgebaut. Kapitel \ref{sec-2} erläutert notwendige Hintergrundinformationen zur Technik der HoloLens, deren Anwendung in der Lehre sowie die physikalischen Hintergründe des Versuches. Kapitel \ref{sec-3} kristallisiert die Anforderungen für den konkreten Anwendungsfall heraus und präzisiert die Problemstellung. In Abschnitt \ref{sec-4} werden die gewählten Lösungsansätze vorgestellt, deren Implementierung in Kapitel \ref{sec-5} erörtert wird. Kapitel \ref{sec-6} stellt die Ergebnisse heraus und diskutiert diese im Rahmen der Fragestellung. Abschließend fasst Kapitel \ref{sec-7} die Ergebnisse zusammen und gibt einen Ausblick.