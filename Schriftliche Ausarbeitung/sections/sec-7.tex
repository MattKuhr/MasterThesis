\section{Zusammenfassung und Ausblick}
\label{sec-7}

\subsection{Zusammenfassung}
TODO...\\
Die Arbeit stellt eine Lösung mit der HoloLens vor, die ein konkretes physikalisches Experiment mit einer Helmholtz-Spule durch eine AR-Anwendung unterstützt. Anhand von Positions- und Echtzeitdaten wird eine Darstellung des Magnetfeldes in den Versuchsaufbau eingebettet. Außerdem werden zusätzliche Informationen und theoretische Ergebnisse angezeigt. Die AR-Anwendung unterstützt das Experiment durch die Visualisierung andernfalls nicht sichtbarer physikalischer Eigenschaften und Zusammenhänge. Gleichzeitig berücksichtigt die Lösung technische Limitierungen der HoloLens. Die Hologramme weisen trotz der unmittelbaren Nähe zu realen Objekten eine ausreichende Stabilität auf und werden durch das begrenzte Field of View der HoloLens kaum beeinträchtigt. Außerdem vermeidet die Lösung Probleme durch zu nah positionierte Objekte und ermöglicht die Nutzung auf für unerfahrene Anwender. Allerdings ist die Übertragbarkeit der Lösung dadurch eingeschränkt, dass sie speziell an die Gegebenheiten wie Größe und Aufbau des Experimentes angepasst ist. Andere Versuche mit anderen Eigenschaften erfordern ggf. andere Maßnahmen, um die HoloLens auch dort einsetzen zu können.

\subsection{Ausblick}
TODO...\\
Empirische Evaluation wäre interessant. Erweiterung der Anwendung um mehr Inhalte ebenfalls. Z.B. Aufbau des Feldes durch Überlagerung der beiden einzelnen Felder. Übertragung auf ein anderes Beispiel auch. Komplexere Anwendungen interessant, z.B. Dipol. Mit HoloLens 2 natürlich, 52° Diagonales Sichtfeld und 2k Auflösung pro Auge