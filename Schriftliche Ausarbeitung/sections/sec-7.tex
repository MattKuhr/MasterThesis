\section{Zusammenfassung und Ausblick}
\label{sec-7}

\subsection{Zusammenfassung}
Die Arbeit stellt eine Lösung mit der HoloLens vor, die ein konkretes physikalisches Experiment mit einer Helmholtz-Spule durch eine AR-Anwendung unterstützt. Anhand von Positions- und Echtzeitdaten wird eine Darstellung des Magnetfeldes in den Versuchsaufbau eingebettet. Außerdem werden theoretische Ergebnisse und zusätzliche Informationen (z.B. Stromrichtung) angezeigt. Die AR-Anwendung unterstützt das Experiment durch die Visualisierung andernfalls nicht sichtbarer physikalischer Eigenschaften und Zusammenhänge.\\
\noindent\hspace*{5mm}
Gleichzeitig berücksichtigt die Lösung technische Limitierungen der HoloLens. Die Hologramme weisen trotz der unmittelbaren Nähe zu realen Objekten eine ausreichende Stabilität auf und werden durch das begrenzte Field of View der HoloLens kaum beeinträchtigt. Außerdem vermeidet die Lösung Probleme durch zu nah positionierte Objekte und ermöglicht die Nutzung auf für unerfahrene Anwender. Jedoch tritt bei einigen Elementen trotz Kantenglättung deutlich sichtbares Kantenflimmern auf. Dennoch waren erste Reaktionen von Nutzern durchgehend sehr positiv.\\
\noindent\hspace*{5mm}
Die Ergebnisse motivieren eine Ausweitung und Übertragung des Ansatzes auf weitere Inhalte und Anwendungsfälle. Allerdings ist die Übertragbarkeit der Lösung dadurch eingeschränkt, dass sie speziell an die Gegebenheiten wie Größe und Aufbau des Experimentes angepasst ist. Andere Versuche mit anderen Eigenschaften erfordern ggf. andere Maßnahmen, um die HoloLens auch dort einsetzen zu können.

\subsection{Ausblick}
Mit der HoloLens 2 wurde kurz vor der Fertigstellung dieser Arbeit der Nachfolger der HoloLens vorgestellt. Diese ist der vorigen Generation technisch in vielen Punkten deutlich überlegen. Die Auflösung pro Auge steigt um das Vierfache und das Sichtfeld vergrößert sich um den Faktor 2,5. Verbesserungen wurden auch bei dem Tragekomfort und dem Interaktionsmodell vorgenommen. Für letzteres unterstützt die neue Brille ein verbessertes Hand-Tracking, bei dem alle zehn Finder individuell erkannt werden.\\

Eine Portierung und Anpassung der Lösung für die HoloLens 2 könnte von diesen Vorteilen profitieren. Das vergrößerte Sichtfeld bietet mehr Freiheiten für weitere Darstellungen und mehr Kopfbewegungen des Nutzers. Mit der gestiegenen Rechenleistung und Auflösung sind außerdem Verbesserungen der Darstellungsqualität möglich, z.B. im Hinblick auf Aliasing und Kantenflimmern. Auch die Interaktion ließe sich mit dem neuen, erweiterten Interaktionsmodell weiter verbessern. Dadurch könnte auch eine Nutzung ohne den Klicker für Nutzer attraktiv sein.
\noindent\hspace*{5mm}
Eine interessante Erweiterung auf Basis des Finger-Trackings könnte die Erläuterung der Rechte-Hand-Regel sein. Diese dient beispielsweise dazu, die Richtung der Kraft zu bestimmen, die ein Magnetfeld auf ein sich bewegendes Elektron ausübt. Denkbar wäre folglich ein Szenario, in dem die HoloLens 2 dieses Vorgehen anleitet und anhand des Trackings erkennt, wann der Nutzer die Regel richtig anwendet.\\

Die Ergebnisse dieser Arbeit motivieren eine Übertragung der Lösung auf die neue Brille zusätzlich. Da die Resultate mit der verwendeten Hard- und Software bereits sehr positiv waren ist davon auszugehen, dass mit der verbesserten Technik noch bessere und umfangreichere Lösungen möglich sind.\\

Neben diesen, durch die technischen Fortschritte geschaffenen Möglichkeiten, wäre auch eine empirische Evaluation der Auswirkungen auf das Lernverhalten durch die Anwendung interessant. Dabei stellt sich die Frage, ob die bei anderen AR-Anwendungen festgestellten Effekte z.B. bezüglich Lernleistung oder Motivation auch bei dieser Umsetzung zu beobachten sind und wie sich die Anwendung diesbezüglich in das Spektrum anderer Lösungen einordnet. 

%Diese ließe sich in zwei Richtungen vornehmen. Zum einen könnte untersucht werden,  Zum anderen wäre eine Erhebung im Hinblick auf die Nutzererfahrung möglich.\\
%\noindent\hspace*{5mm}
%Mit ersterem Ansatz 
