\section*{Zusammenfassung}
Mit Augmented Reality (AR) hat in den letzten Jahren durch die Verfügbarkeit entsprechender Hard- und Software an Bedeutung gewonnen. Die Technik erlaubt es, virtuelle Objekte in die Realität einzubetten und kommt unter anderem im Ausbildungsbereich zum Einsatz. Auch in Bereich der physikalischen Ausbildung wurde AR bereits erfolgreich eingesetzt. 

Anwendung findet AR unter anderem auch in der Ausbildung. Mit der HoloLens als AR-Device sind solch immersive AR-Anwendungen umsetzbar. 

Die Arbeit stellt eine Lösung vor, in der ein konkretes physikalisches Experiment mit einer Helmholtz-Spule durch eine AR-Anwendung unterstützt. Anhand von Positions- und Echtzeitdaten wird eine Darstellung des Magnetfeldes in den Versuchsaufbau eingebettet. Außerdem werden theoretische Ergebnisse und zusätzliche Informationen (z.B. Stromrichtung) angezeigt. Dadurch werden andernfalls nicht sichtbarer physikalische Eigenschaften und Zusammenhänge für den Nutzer sichtbar.\\

\noindent\hspace*{5mm}
Dabei berücksichtigt die Lösung technische Limitierungen der HoloLens und zielt darauf ab, bekannte Probleme wie instabile oder abgeschnittene Objekte zu vermeiden. Im Ergebnis Die Hologramme weisen eine gute Stabilität auf und werden durch das begrenzte Field of View der HoloLens kaum beeinträchtigt. Jedoch tritt bei einigen Elementen trotz Kantenglättung deutlich sichtbares Kantenflimmern auf. Dennoch waren erste Reaktionen von Nutzern durchgehend sehr positiv.\\

\noindent\hspace*{5mm}
Die Ergebnisse motivieren eine Ausweitung und Übertragung des Ansatzes auf weitere Inhalte und Anwendungsfälle. Allerdings ist die Übertragbarkeit der Lösung dadurch eingeschränkt, dass sie speziell an die Gegebenheiten wie Größe und Aufbau des Experimentes angepasst ist. Andere Versuche mit anderen Eigenschaften erfordern ggf. andere Maßnahmen, um die HoloLens auch dort einsetzen zu können.

\section*{Abstract}
AR -> HoloLens -> Anwendung Physik -> Problem konkreter Versuch -> Lösungsansatz -> Ergebnisse -> Ausblick