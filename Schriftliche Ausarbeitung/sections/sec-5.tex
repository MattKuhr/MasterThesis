\section{Das Wie der Umsetzung}
\label{sec-5}
\fbox{
\parbox{\linewidth}{
	\textit{Ziel des Kapitels:}\\
	Umsetzung erläutern: Wie wurde die Lösungsidee umgesetzt?\\[6px]
	\textit{Inhalte:}
	\begin{itemize}
		\item Welche Darstellungen wurden gewählt?
		\item Wie wurde die Interaktion gestaltet?
	\end{itemize}
}}
\subsection{Architektur der Anwendung}
%TODO Komponentendiagramm erstellen
\begin{itemize}
	\item Blender, Unity, Vuforia
	\item Übersicht Komponenten und Zusammenspiel
\end{itemize}

\textbf{Client-Server Datenübertragung}
%TODO Sequenzdiagramm erstellen
\begin{itemize}
	\item HoloLens erfragt Daten vom Server über HTTP GET-Requests
	\item Asynchrone Anfragen, so wird Render-Prozess nicht blockiert
	\item Server nimmt Request entgegen und hält eine Antwort solange zurück, bis er einen neuen Wert vom Arduino erhält, der der HoloLens noch nicht bekannt ist und von vorherigen Werten signifikant abweicht
	\item Dadurch kommen neue Daten sehr schnell bei der HoloLens an, der Nutzer sieht die Änderungen auf der Brille sofort, wenn er Änderungen an der Spannungsquelle vornimmt
	\item Dadurch wird zeitliche Einbettung und Kontinuität erreicht
	\item Dabei wird jedoch kein unnötiger Traffic erzeugt, wenn es keine Änderungen gibt
	\item Das spart Ressourcen
\end{itemize}


\subsection{Darstellung}
\begin{itemize}
	\item Magnetfeld Darstellung mit Vorberechnung durch XYZ
	\item Depth Cues für 3D-Objekte
	\subitem Spulen virtuell nachmodelliert und für Occlusion Berechnung eingefügt
	\subitem Drop Shadows für 3D Pfeile entwickelt
	\item Informationen zum Versuchsaufbau via ToolTips, optional einblendbar
	\item ...
\end{itemize}

\textbf{Occlusion Berechnung}
\begin{itemize}
	\item Für Verdeckung werden maßstabsgetreu nachmodellierte, virtuelle Objekte verwendet
	\item 3D Mesh in Blender erstellt, 2mm größer als echte Objekte für Spielraum
	\item Objekte werden möglichst genau über reale gelegt
	\item Rendering erfolgt ausschließlich in den Z-Puffer, dadurch sind die realen Objekte sichtbar, die virtuellen verdecken jedoch dahinterliegende, vrituelle Objekte
	\item Das Near Clipping Plane muss dafür jedoch sehr nah am Kameraursprung liegen, andernfalls würden weiter entfernte, virtuelle Objekte plötzlich doch vor realen Objekten angezeigt werden, sobald letztere zu nah sind und das Clipping die Objekte zur Verdeckungsberechnung vom Rendering ausschließt
\end{itemize}

\subsection{Interaktion}
\begin{itemize}
	\item Per Doppelklick Geste zwischen verschiedenen Modi umschalten
	\item Per Tastatur Eingabe von numerischen Werten
	\item Per Keywords "Power on" und "Power off" Steuerung des Stromkreises
	\item Klick-Feedback durch Sound
	\item ...
\end{itemize}
