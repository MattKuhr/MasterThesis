\part{Ziel \& Frage}
\label{part:goal}
\begin{frame}[fragile]{}
\usebeamerfont{frametitle}\textcolor{blue}{Ziel:} \usebeamerfont{text}HoloLens einsetzen, um Versuchsaufbau mit Informationen anzureichern.
\pause
\vskip 1em
\usebeamerfont{frametitle}\textcolor{blue}{Frage:} \usebeamerfont{text}Wie kann die HoloLens für diesen Versuch konkret eingesetzt werden?
\end{frame}

\part{Anforderungen}
\begin{frame}[fragile]{}
\begin{center}
\includegraphics[width=0.9\textwidth]{images/Anforderungen.png}
\end{center}
\end{frame}

\part{Physikalische Seite}
\begin{frame}[fragile]{Anforderungen aus dem Anwendungsfall heraus}
\begin{itemize}
%	\setlength{\itemsep}{0.01\baselineskip}
\item Magnetfeld von Erde und Spule
\begin{itemize}
%		\setlength{\itemsep}{0.01\baselineskip}
\item Stärke
\item Richtung
\item Homogenität
\item Inhomogenität am Rand der Spule andeuten (Optional) 
\end{itemize}
\item Stromfluss durch die Spule
\begin{itemize}
\item Richtung
\item Kennzeichnung von Plus und Minus
\item Stärke (Optional) 
\end{itemize}
\end{itemize}
\end{frame}

\begin{frame}[fragile]{Anforderungen aus dem Anwendungsbereich}
\begin{itemize}
\setlength{\itemsep}{-0.25em}
\item Kompass
\begin{itemize}
\setlength{\itemsep}{-0.25em}
\item Nordrichtung
\item Grobe Auslenkung der Nadel
\end{itemize}
\item Weitere Informationen (Optional)
\begin{itemize}
\item Windungszahl der Spule
\item Durchmesser und Abstand der Spulen
\item Numerische Werte und Informationen (z.B. Fließt aktuell Strom, angelegte Stromstärke, angenommene Stärke des Erdmagnetfeldes, systematischer und zufälliger Fehler, etc.)
\end{itemize}
\end{itemize}
\end{frame}

\part{Technische Seite}
\label{part:tech}

\begin{frame}[fragile]{Eigenschaften der HoloLens}
\begin{itemize}
\item \textit{See-Through Display}, stereoskopisch, Zwei 16:9 HD Bilder, 32\degree Field of View, \textit{Color Sequential Display}, 60 Hz Bildwiederholrate hochskaliert auf 240 Hz, Akkomodation der Augen fest bei 2 m
\pause
\begin{itemize}
\item Für Objekte: Größe, Geschwindigkeit, Farbe, Distanz zur Kamera
\pause
\item Schwarz = Transparent, kein fester Hintergrund, Darstellungen überdecken Hintergrund, Helligkeit im Raum beeinflusst Darstellung
\end{itemize}
\pause
\item \textit{Inside-Out Tracking} über Tiefenkamera (Infrarot), Stereo-Kameras und Inertialmesssystem (IMU)
\begin{itemize}
\pause
\item 60 FPS stabil halten, stark spiegelnde oder transparente Oberflächen vermeiden, mögliche Einflüsse auf die Sensoren beachten
\end{itemize}
\end{itemize}
\end{frame}

\begin{frame}[fragile]{Eigenschaften der HoloLens}
\begin{itemize}
\item Stand-Alone Device, 1 GHz CPU, 1 GHz HPU, 2 GB RAM, 1 GB HPU RAM, passiv gekühlt
\begin{itemize}
\pause
\item Strenges Performance Limit, keine rechenintensiven Anwendungen möglich, Akkulaufzeit, Qualitätseinschränkungen bei Mixed Reality Capture, etc.
\end{itemize}
\pause
\item Interaktion über Handgesten und Sprache, Cursor = vorwärtsgerichteter Raycast im Zentrum des Sichtfeldes
\item Usability und UX Einschränkungen bzw. Empfehlungen
\end{itemize}
\end{frame}


\part{Problemstellung}
\label{part:golas}
\begin{frame}
\vspace{-1em}
\begin{center}
\includegraphics[width=0.8\textwidth]{images/Informiertes_Design.png}	
\end{center}
\usebeamerfont{frametitle}\textcolor{blue}{Fragen:}
\begin{itemize}
\item Was soll dargestellt werden?
\item Wie soll es dargestellt werden?
\item Wie soll damit interagiert werden?
\end{itemize}
\vspace{50px}
\end{frame}
