\part{Einleitung}
\label{part:intro}

\begin{frame}[fragile]{}
Einleitung!
\end{frame}

\part{HoloLens}
\label{part:hololens}
\begin{frame}[fragile]{}
\begin{figure}[h!]
	\centering
	\includegraphics[width=0.7\textwidth]{images/hololens.jpg}
\end{figure}
\begin{itemize}
	\pause
	\item \textit{Mixed Reality} Device
	\pause
	\item Projeziert virtuelle Objekte in das Sichtfeld des Nutzers
	\pause
	\item Nutzer bewegt sich simultan durch reale und virtuelle Szene
	\pause
	\item Genaue Bestimmung von Position und Ausrichtung im Raum durch Sensoren: \textit{Inside-Out-Tracking}
	\pause
	\item Interaktion über Gesten und Sprache
\end{itemize}	
\end{frame}

\begin{frame}[fragile]{}
\begin{figure}[h!]
	\centering
	\includegraphics[width=0.5\textwidth]{images/hololens.jpg}
\end{figure}
\begin{itemize}
	\pause
	\item \textit{See-Through, Color Sequential} Display, stereoskopisch, Zwei 16:9 HD Bilder, 32\degree Field of View, 60 Hz Bildwiederholrate hochskaliert auf 240 Hz, Akkomodation der Augen fest bei 2 m
	\pause
	\item \textit{Inside-Out Tracking} über Tiefenkamera (Infrarot), Stereo-Kameras und Inertialmesssystem (IMU)
	\pause
	\item Stand-Alone Device, 1 GHz CPU, 1 GHz HPU, 2 GB RAM, 1 GB HPU RAM, passiv gekühlt, Bluetooth, WiFi
	\pause
	\item Interaktion über Handgesten und englische Sprachkommandos, Cursor = vorwärtsgerichteter Raycast im Zentrum des Sichtfeldes
\end{itemize}	
\end{frame}

\part{State of the Art}
\label{part:sota}
\begin{frame}[fragile]{HoloLens in der Physik}
\begin{figure}
	\includegraphics[width=0.43\textwidth]{images/Amiraslanov18.png}
	\includegraphics[width=0.43\textwidth]{images/Javaheri18.png}
	\begin{center}
	\vspace{0.05cm}
	\includegraphics[width=0.87\textwidth]{images/Strzys18.png}	
	\end{center}
	\setlength{\abovecaptionskip}{7pt plus 5pt minus 2pt}
	\caption*{Oben Links: \citep{Amiraslanov18}, Oben Rechts: \cite{Javaheri18}, Unten: \cite{Strzys18}}
\end{figure}
\end{frame}

\begin{frame}[fragile]{Mixed Reality in der Physik}

\begin{figure}
	\includegraphics[width=0.35\textwidth]{images/Buchau09.jpg}
	\hspace{0.05cm}
	\includegraphics[width=0.35\textwidth]{images/Matsutomo13.jpg}

%	\centering
	\includegraphics[width=0.35\textwidth]{images/Buchau09_Magnet.jpg}
	\hspace{0.05cm}
	\includegraphics[width=0.35\textwidth]{images/Ibanez14.jpg}

	\setlength{\abovecaptionskip}{5pt plus 5pt minus 2pt}
	\caption*{Links: \citep{Buchau09}, Rechts Oben: \cite{Matsutomo13}, Rechts Unten: \cite{Ibanez14}}
\end{figure}
\end{frame}

\part{Anwendungsfall: Helmholtz-Spulen}
\label{part:physics}
\begin{frame}[fragile]{Experiment: Bestimmung des Erdmagnetfeldes}
\begin{minipage}{0.5\textwidth}
	{\setstretch{1.0}
\begin{itemize}[itemsep=1mm]
	\item[$1.$] Kompass ausrichten lassen
	\item[$2.$] Spule orthogonal zur Nord-Süd-Achse aufstellen
	\item[$3.$] Stromfluss erhöhen, bis Kompassnadel um 45\degree ausgelenkt ist
	\item[$4.$] Flussdichte mit Formel aus Stromstärke und Spuleneigenschaften berechnen
\end{itemize}
}
\end{minipage}
\begin{minipage}{0.48\textwidth}
	\centering
	\includegraphics[width=\textwidth]{images/Magnetfeld-Helmholtzspule.jpg}
	\tiny Magnetfeld einer Helmholtz-Spule in der X-Z-Ebene. Überlagerte Darstellung von Vektorfeld und Feldlinien. Quelle: Wikipedia.
\end{minipage}
\end{frame}
