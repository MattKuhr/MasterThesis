\documentclass{beamer}
\providecommand\insertframetitle{}

\usepackage{ngerman}
\usepackage{environ}
\usepackage{graphicx}
\usepackage[tikz]{bclogo}
\usepackage{tikz}
\usepackage{listings}
\usepackage{multimedia}
\usepackage{hyperref}
\usepackage{array,multirow,multicol}
\usetikzlibrary{calc}
\usepackage[utf8]{inputenc}
\usepackage[overlay,absolute]{textpos}
\usepackage{xpatch}
\usepackage{booktabs}
\usepackage{array}
\usepackage{wrapfig}
\usepackage[font=scriptsize,labelfont=bf]{caption}
\usepackage{fancyhdr}
\usepackage{enumerate}
\usepackage{enumitem}
\usepackage{gensymb}
\usepackage{setspace}

\setlist[itemize,1]{label=$\bullet$, itemsep=-1mm}
\setlist[itemize,2]{label=$\diamond$, topsep=0pt, itemsep=-1mm}



\usepackage[style=alphabetic,sortlocale=de_DE,url=false,natbib=true,backend=bibtex]{biblatex}
\beamertemplatenavigationsymbolsempty
\makeatletter
\xpatchcmd{\beamer@part}{\Hy@writebookmark{\the\c@section}{#1}{Outline\the\c@part}{1}{toc}}{}{}{}
\newcommand{\srcsize}{\@setfontsize{\srcsize}{5pt}{5pt}}
\usepackage[uni, footuni, headframelogo]{./unirostock/beamerthemeRostock}
\usepackage{graphicx}
\usepackage{multimedia}
\footinstitute{Institut für Informatik}

\addbibresource{bibliography.bib}

\title{Entwicklung von 3D-Darstellungen mit der HoloLens zur Unterstützung der Vermittlung physikalischer Inhalte}
\subtitle{Zwischenverteidigung der Masterarbeit}
\author{Matthias Kuhr}
\date{\today}

\makeatletter
\makeatother
\headLogoTop{standard} % Main-Logo (``hafen'' or ``standard'')
%\secondHeadLogoTop{20}{dbis_logo.jpg} % Optional: Image for a second logo in the header
%\showNaviSymbols % show Navi-symbols in the footer
\showNaviDots %show Navi-Dots in the header
\withParts %toogle Pagenumbering with parts on
%\printmode{handout} % Remove decoration
\bibliography{bibliography.bib}

\begin{document}
{
	\setbeamertemplate{footline}[titlewithoutnumber] %hg%no numbers on first page
	\maketitle
	
	\nocite{Zimmer17,Milgram94}
}

%{ 
%\setbeamertemplate{footline}[titlewithoutnumber] %hg%no numbers on first page
%\begin{frame}{Gliederung}
%\begin{enumerate}
%	\item Einleitung und Zielstellung
%	\item Technischer Hintergrund
%	\item Strategien und Lösungsansätze für das Verteilungsproblem
%	\item Ergebnisse einer praktischen Umsetzung
%	\item Zusammenfassung und Ausblick
%\end{enumerate}
%\end{frame}
%}

%\part{Einleitung}
\label{part:intro}

\begin{frame}[fragile]{}
Einleitung!
\end{frame}

\part{HoloLens}
\label{part:hololens}
\begin{frame}[fragile]{}
\begin{figure}[h!]
	\centering
	\includegraphics[width=0.7\textwidth]{images/hololens.jpg}
\end{figure}
\begin{itemize}
	\pause
	\item \textit{Mixed Reality} Device
	\pause
	\item Projeziert virtuelle Objekte in das Sichtfeld des Nutzers
	\pause
	\item Nutzer bewegt sich simultan durch reale und virtuelle Szene
	\pause
	\item Genaue Bestimmung von Position und Ausrichtung im Raum durch Sensoren: \textit{Inside-Out-Tracking}
\end{itemize}	
\end{frame}

\part{State of the Art}
\label{part:sota}

\begin{frame}[fragile]{HoloLens in der Physik}
%\begin{itemize}
%	\item 
%	\item Darstellung des Wärmeprofiles bei einem erhitzten Metallstab \cite{Strzys18}	
%\end{itemize}

\begin{figure}
\includegraphics[width=0.45\textwidth]{images/Amiraslanov18.png}
\includegraphics[width=0.45\textwidth]{images/Javaheri18.png}
\begin{center}
\vspace{0.05cm}
\includegraphics[width=0.9\textwidth]{images/Strzys18.png}	
\end{center}
\setlength{\abovecaptionskip}{5pt plus 5pt minus 2pt}
\caption*{Oben Links: \citep{Amiraslanov18}, Oben Rechts: \cite{Javaheri18}, Unten: \cite{Strzys18}}
\end{figure}

\end{frame}

\begin{frame}[fragile]{Mixed Reality in der Physik}

\begin{figure}
	\includegraphics[width=0.35\textwidth]{images/Buchau09.jpg}
	\hspace{0.05cm}
	\includegraphics[width=0.35\textwidth]{images/Matsutomo13.jpg}

%	\centering
	\includegraphics[width=0.35\textwidth]{images/Buchau09_Magnet.jpg}
	\hspace{0.05cm}
	\includegraphics[width=0.35\textwidth]{images/Ibanez14.jpg}

	\setlength{\abovecaptionskip}{5pt plus 5pt minus 2pt}
	\caption*{Links: \citep{Buchau09}, Rechts Oben: \cite{Matsutomo13}, Rechts Unten: \cite{Ibanez14}}
\end{figure}

\end{frame}

\part{Problemstellung und Ziele}
\label{part:golas}

\begin{frame}[fragile]{}
Probleme und Ziele!
\end{frame}

\part{Design}
\label{part:design}

\begin{frame}[fragile]{}
Design
\end{frame}

\part{Umsetzung und Ausblick}
\label{part:practice}

\begin{frame}[fragile]{}
ASDF
\end{frame}
\part{Hintergrund und Motivation}
\label{part:intro}

\begin{frame}[fragile]{Motivation}
\vspace{10px}
\usebeamerfont{frametitle}\textcolor{blue}{Frage:} \usebeamerfont{text}
\end{frame}

\begin{frame}[fragile]{Struktur des Vortrages}
\begin{itemize}
	\item Hintergründe zur HoloLens, deren Einsatz in der Physik und zum gewählten Experiment
	\item Probleme \& Anforderungen
	\item Lösungsansatz
	\item Umsetzung \& Ergebnisse
	\item Diskussion
\end{itemize}
\end{frame}

\part{HoloLens}
\label{part:hololens}
\begin{frame}[fragile]{}
\begin{figure}[h!]
	\centering
	\includegraphics[width=0.7\textwidth]{images/papers/hololens.jpg}
\end{figure}
\begin{itemize}
	\item Projeziert virtuelle Objekte in das Sichtfeld des Nutzers
	\item Nutzer bewegt sich simultan durch reale und virtuelle Szene
	\item Genaue Bestimmung von Position und Ausrichtung im Raum durch Sensoren: \textit{Inside-Out-Tracking}
	\item Interaktion über Gesten und Sprache, vorgefertigte Objekte und Funktionen
	\item Ermöglicht \textit{Augmented Reality} Anwendungen
\end{itemize}	
\end{frame}

\begin{frame}[fragile]{Technische Aspekte}
		\begin{itemize}
			\item \textit{See-Through Display}, Zwei 16:9 HD Bilder, 32\degree FoV (diagonal)
			\begin{itemize}[topsep=-5px]
				\setlength{\itemsep}{-5px}
				\item Einfluss auf Größe und Farbe von Objekten
			\end{itemize}
			\pause
			\item Akkomodation der Augen fest bei 2 m
			\begin{itemize}[topsep=-5px]
				\setlength{\itemsep}{-5px}
				\item Einfluss auf Distanz
			\end{itemize}
			\item \textit{Inside-Out Tracking} über Tiefenkamera, Stereo-Kameras und IMU
			\begin{itemize}[topsep=-5px]
				\setlength{\itemsep}{-5px}
				\item Einfluss auf Positionierung und Stabilität
			\end{itemize}
			\pause
			\item Stand-Alone Device, 1 GHz CPU/HPU, 2 GB RAM
			\begin{itemize}[topsep=-5px]
				\setlength{\itemsep}{-5px}
				\item Einfluss auf Performance
			\end{itemize}
			\item Optimiert Stabilität der Objekte für ausgewählte Ebene
			\begin{itemize}[topsep=-5px]
				\setlength{\itemsep}{-5px}
				\item Einfluss auf Positionierung und Stabilität
			\end{itemize}
		\end{itemize}
\end{frame}

\part{State of the Art}
\label{part:sota}
\begin{frame}[fragile]{HoloLens in der Physik}
\vspace{0.1cm}
\begin{figure}
	\includegraphics[width=0.42\textwidth]{images/papers/Amiraslanov18.png}
	\includegraphics[width=0.42\textwidth]{images/papers/Javaheri18.png}
	\begin{center}
	\vspace{0.03cm}
	\includegraphics[width=0.85\textwidth]{images/papers/Strzys18.png}	
	\end{center}
	\setlength{\abovecaptionskip}{7pt plus 5pt minus 2pt}
	\caption*{Oben Links: Amiraslanov (2018), Oben Rechts: Javaheri (2018), Unten: Strzys (2018)}
\end{figure}
\end{frame}

\begin{frame}[fragile]{Augmented Reality in der Physik}
	\vspace{0.1cm}
\begin{figure}
	\includegraphics[width=0.32\textwidth]{images/papers/Buchau09.jpg}
	\hspace{0.05cm}
	\includegraphics[width=0.32\textwidth]{images/papers/Matsutomo13.jpg}

%	\centering
	\includegraphics[width=0.32\textwidth]{images/papers/Buchau09_Magnet.jpg}
	\hspace{0.05cm}
	\includegraphics[width=0.32\textwidth]{images/papers/Ibanez14.jpg}

	\setlength{\abovecaptionskip}{7pt plus 5pt minus 2pt}
	\caption*{Links: Buchau (2009), Rechts Oben: Matsutomo (2013), Rechts Unten: Ibanez (2014)}
\end{figure}
\end{frame}

\part{Anwendungsfall: Helmholtz-Spulen}
\label{part:physics}
\begin{frame}[fragile]{Physikalischer Hintergrund}
\begin{minipage}{0.5\textwidth}
	{\setstretch{1.0}
		\begin{itemize}[itemsep=1mm]
			\item Magnetfeld ist 3D-Vektorfeld, Flussdichte $\vec{B}$ in Tesla
			\item Stromfluss durch Spule erzeugt ein Magnetfeld, abhängig von Stromstärke
			\item Helmholtz-Spule: Feld im Inneren weitgehend \textit{homogen}
			\item Feldlinien und Vektormodell sind etablierte Darstellungsmodelle
		\end{itemize}
	}
\end{minipage}
\begin{minipage}{0.45\textwidth}
	\centering
	\includegraphics[width=0.9\textwidth]{images/papers/hh_mfield_nocol.png}\\
	\small Darstellung des Magnetfeldes einer Helmholtz-Spule mittels Feldlinien für eine Ebene.
\end{minipage}
\end{frame}

\begin{frame}[fragile]{Experiment: Bestimmung des Erdmagnetfeldes}
\begin{minipage}{0.5\textwidth}
	{\setstretch{1.0}
\begin{itemize}[itemsep=1mm]
	\item[$1.$] Kompass ausrichten lassen
	\item[$2.$] Spule orthogonal zur Nord-Süd-Achse aufstellen
	\item[$3.$] Spannungsquelle einschalten und Stromfluss erhöhen, bis Kompassnadel um 45\degree ausgelenkt ist
	\item[$4.$] Flussdichte mit Formel aus Stromstärke und Spuleneigenschaften berechnen
\end{itemize}
}
\end{minipage}
\begin{minipage}{0.45\textwidth}
	\centering
	\includegraphics[width=0.9\textwidth]{images/papers/setup_labled.jpg}\\
	\small Foto des Versuchsaufbaus mit Bezeichnung der Elemente.
\end{minipage}
\end{frame}

\part{Ziel \& Frage}
\label{part:goal}
\begin{frame}[fragile]{}
\usebeamerfont{frametitle}\textcolor{blue}{Ziel:} \usebeamerfont{text}HoloLens einsetzen, um Versuchsaufbau mit Informationen anzureichern.
\pause
\vskip 1em
\usebeamerfont{frametitle}\textcolor{blue}{Frage:} \usebeamerfont{text}Wie kann die HoloLens für diesen Versuch konkret eingesetzt werden?
\vskip 0.5em
\begin{itemize}
	\item Was soll mit der HoloLens dargestellt werden?
	\item Wie soll diese Darstellung erfolgen?
	\item Und wie soll mit den dargestellten Informationen interagiert werden?
\end{itemize}
\vspace{50px}
\end{frame}
\begin{comment}

\begin{frame}[fragile]{Inhaltliche Anforderungen}
\textit{Was soll mit der HoloLens dargestellt werden?}
\pause
\begin{itemize}
	\item Zusammenhänge zwischen Magnetfeldern, Stromfluss und Kompass
\end{itemize}
\pause
\textit{Wie soll diese Darstellung erfolgen?}
\begin{itemize}
	\item Physikalische Eigenschaften qualitativ und als solche interpretierbar wiedergeben
	\item Interaktion mit dem Versuch soll nicht eingeschränkt werden
\end{itemize}
\end{frame}


\begin{frame}[fragile]{Anforderungen aus technischen Gegebenheiten}
\textit{Wie soll diese Darstellung erfolgen und wie soll mit den dargestellten Informationen interagiert werden?}
\pause
\begin{itemize}
\item Größe, Geschwindigkeit, Farbe, Distanz zur Kamera von Objekten
\pause
\item Korrekte Positionierung und Stabilität der Hologramme gewährleisten
\pause
\item Stabile Framerate von 60 FPS erreichen
\pause
\item Empfehlungen zu Usability und UX beachten
\end{itemize}
\end{frame}
\end{comment}

\part{Lösungsansatz}
\label{part:solution}
\begin{frame}[fragile]{}
	\textit{Erweiternde Darstellungen erstellen und räumlich wie zeitlich in den Versuch integrieren}
	\pause
	\begin{itemize}
		\item Magnetfelder, deren Eigenschaften, Stromfluss und Auswirkung auf die Nadel anzeigen
		\item Positionierung und Stabilisierung der HoloLens nutzen		
		\item Echtzeitdaten (Messwerte) an die HoloLens übermitteln und darstellen
		\item Technische Einschränkungen beim Design berücksichtigen
	%	\begin{itemize}[topsep=-5px]
	%		\setlength{\itemsep}{-5px}
	%		\item Maßnahmen zur Vermeidung von Problemen anwenden
	%		\item Angepasstes Design
	%		\item Vorgefertigte Objekte nutzen
	%	\end{itemize}
		\item Anpassung von realen Objekten, um Integration mit der Anwendung zu verbessern
	\end{itemize}
\end{frame}

\part{Lösung}
\begin{frame}[fragile]{}
	\vspace{-10px}
	\centering
	\includegraphics[width=0.75\textwidth]{images/unity/overview.jpg}\\
	\scriptsize Darstellungen in der Entwicklungsumgebung (Unity). Auflösung und Qualitätseinstellungen entsprechen den Werten auf der HoloLens.
\end{frame}

\begin{frame}[fragile]{}
\begin{figure}
	\vspace{-10px}
	\centering
	\includegraphics[width=0.7\textwidth]{images/HL/fieldlines_cut.jpg}\\
	\scriptsize Screenshot von der HoloLens mit Feldliniendarstellung.
\end{figure}
\end{frame}

\begin{frame}[fragile]{}
\begin{figure}
	\vspace{-10px}
	\centering
	\includegraphics[width=0.85\textwidth]{images/HL/Vektoren.jpg}\\
	\scriptsize Screenshot von der HoloLens mit Vektordarstellung.
\end{figure}
\end{frame}

\begin{frame}[fragile]{Near Plane Fading}
\begin{figure}
	\includegraphics[width=0.8\textwidth]{images/HL/compass.jpg}
\end{figure}
\end{frame}

\begin{frame}[fragile]{Design}
\begin{itemize}
	\item Nutzung etablierter Darstellungsmodelle
	\pause
	\item Anwendung für Nutzung im Abstand von ca. 1,3 m designet, zu nah liegende Objekte werden ausgeblendet
	\begin{itemize}
		\item Hologramme passen ins Sichtfeld, sind nicht zu dicht positioniert, nutzen Screenspace aus, Komfortable Nutzung möglich
	\end{itemize}
	\pause
	\item 
	\pause
	\item Umsetzung empfohlener Maßnahmen zur Verbesserung der Performance
\end{itemize}
\end{frame}


\part{Ergebnisse}
\label{part:results}
\begin{frame}[fragile]{Erweiterungen}
\begin{itemize}
	\item Visualisierung der Komponenten des Magnetfeldes in zwei Darstellungen und in Echtzeit
	\item Darstellung einer vorberechneten Lösung für eine ausgewählte Ebene des Feldes der Spule
	\item Kennzeichnung der Stromrichtung
	\item Integration einer virtuellen Kompass-Skala mit Hervorhebung wichtiger Zustände
	\item Einbettung einer virtuellen Kompassnadel auf Basis theoretischer Werte
	\item Numerische Darstellung gemessener und berechneter Echtzeitdaten
\end{itemize}
\end{frame}


\part{Ausblick}
\label{part:future}
\begin{frame}[fragile]{Erweiterungen}
\usebeamerfont{frametitle}\textcolor{blue}{Inhaltlich:} \usebeamerfont{text}\textit{Weitere Lerninhalte integrieren}
\begin{itemize}
	\item Weitere Inhalte, z.B. Rechte-Hand-Regel
	\item Weitere Experimente, z.B. Ablenkung eines Elektronenstrahles
\end{itemize}
\pause
\vskip 1em
\usebeamerfont{frametitle}\textcolor{blue}{Technische:} \usebeamerfont{text}\textit{Portierung für HoloLens 2}
\begin{itemize}
	\item Auflösung x4 pro Auge, Sichtfeld x2 (Fläche)
	\item Tragekomfort und Interaktion verbessert
\end{itemize}

\vspace{50px}
\end{frame}



\part{Diskussion}
\begin{frame}[fragile]{}
Vielen Dank für Ihre Aufmerksamkeit!

\vspace{1em}
\hspace{1em} Fragen?
\end{frame}


\end{document}
