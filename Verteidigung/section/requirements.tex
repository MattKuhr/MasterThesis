\part{Ziel \& Frage}
\label{part:goal}
\begin{frame}[fragile]{}
\usebeamerfont{frametitle}\textcolor{blue}{Ziel:} \usebeamerfont{text}HoloLens einsetzen, um Versuchsaufbau mit Informationen anzureichern.
\pause
\vskip 1em
\usebeamerfont{frametitle}\textcolor{blue}{Frage:} \usebeamerfont{text}Wie kann die HoloLens für diesen Versuch konkret eingesetzt werden?
\vskip 0.5em
\begin{itemize}
	\item Was soll mit der HoloLens dargestellt werden?
	\item Wie soll diese Darstellung erfolgen?
	\item Und wie soll mit den dargestellten Informationen interagiert werden?
\end{itemize}
\vspace{50px}
\end{frame}
\begin{comment}

\begin{frame}[fragile]{Inhaltliche Anforderungen}
\textit{Was soll mit der HoloLens dargestellt werden?}
\pause
\begin{itemize}
	\item Zusammenhänge zwischen Magnetfeldern, Stromfluss und Kompass
\end{itemize}
\pause
\textit{Wie soll diese Darstellung erfolgen?}
\begin{itemize}
	\item Physikalische Eigenschaften qualitativ und als solche interpretierbar wiedergeben
	\item Interaktion mit dem Versuch soll nicht eingeschränkt werden
\end{itemize}
\end{frame}


\begin{frame}[fragile]{Anforderungen aus technischen Gegebenheiten}
\textit{Wie soll diese Darstellung erfolgen und wie soll mit den dargestellten Informationen interagiert werden?}
\pause
\begin{itemize}
\item Größe, Geschwindigkeit, Farbe, Distanz zur Kamera von Objekten
\pause
\item Korrekte Positionierung und Stabilität der Hologramme gewährleisten
\pause
\item Stabile Framerate von 60 FPS erreichen
\pause
\item Empfehlungen zu Usability und UX beachten
\end{itemize}
\end{frame}
\end{comment}
