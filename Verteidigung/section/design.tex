\part{Lösungsansatz}
\label{part:solution}
\begin{frame}[fragile]{}
	\textit{Erweiternde Darstellungen erstellen und räumlich wie zeitlich in den Versuch integrieren}
	\pause
	\begin{itemize}
		\item Magnetfelder, deren Eigenschaften, Stromfluss und Auswirkung auf die Nadel anzeigen
		\item Positionierung und Stabilisierung der HoloLens nutzen		
		\item Echtzeitdaten (Messwerte) an die HoloLens übermitteln und darstellen
		\item Technische Einschränkungen beim Design berücksichtigen
	%	\begin{itemize}[topsep=-5px]
	%		\setlength{\itemsep}{-5px}
	%		\item Maßnahmen zur Vermeidung von Problemen anwenden
	%		\item Angepasstes Design
	%		\item Vorgefertigte Objekte nutzen
	%	\end{itemize}
		\item Anpassung von realen Objekten, um Integration mit der Anwendung zu verbessern
	\end{itemize}
\end{frame}

\part{Lösung}
\begin{frame}[fragile]{}
	\vspace{-10px}
	\centering
	\includegraphics[width=0.75\textwidth]{images/unity/overview.jpg}\\
	\scriptsize Darstellungen in der Entwicklungsumgebung (Unity). Auflösung und Qualitätseinstellungen entsprechen den Werten auf der HoloLens.
\end{frame}

\begin{frame}[fragile]{}
\begin{figure}
	\vspace{-10px}
	\centering
	\includegraphics[width=0.7\textwidth]{images/HL/fieldlines_cut.jpg}\\
	\scriptsize Screenshot von der HoloLens mit Feldliniendarstellung.
\end{figure}
\end{frame}

\begin{frame}[fragile]{}
\begin{figure}
	\vspace{-10px}
	\centering
	\includegraphics[width=0.85\textwidth]{images/HL/Vektoren.jpg}\\
	\scriptsize Screenshot von der HoloLens mit Vektordarstellung.
\end{figure}
\end{frame}

\begin{frame}[fragile]{Near Plane Fading}
\begin{figure}
	\includegraphics[width=0.8\textwidth]{images/HL/compass.jpg}
\end{figure}
\end{frame}

\begin{frame}[fragile]{Design}
\begin{itemize}
	\item Nutzung etablierter Darstellungsmodelle
	\pause
	\item Anwendung für Nutzung im Abstand von ca. 1,3 m designet, zu nah liegende Objekte werden ausgeblendet
	\begin{itemize}
		\item Hologramme passen ins Sichtfeld, sind nicht zu dicht positioniert, nutzen Screenspace aus, Komfortable Nutzung möglich
	\end{itemize}
	\pause
	\item 
	\pause
	\item Umsetzung empfohlener Maßnahmen zur Verbesserung der Performance
\end{itemize}
\end{frame}


\part{Ergebnisse}
\label{part:results}
\begin{frame}[fragile]{Erweiterungen}
\begin{itemize}
	\item Visualisierung der Komponenten des Magnetfeldes in zwei Darstellungen und in Echtzeit
	\item Darstellung einer vorberechneten Lösung für eine ausgewählte Ebene des Feldes der Spule
	\item Kennzeichnung der Stromrichtung
	\item Integration einer virtuellen Kompass-Skala mit Hervorhebung wichtiger Zustände
	\item Einbettung einer virtuellen Kompassnadel auf Basis theoretischer Werte
	\item Numerische Darstellung gemessener und berechneter Echtzeitdaten
\end{itemize}
\end{frame}


\part{Ausblick}
\label{part:future}
\begin{frame}[fragile]{Erweiterungen}
\usebeamerfont{frametitle}\textcolor{blue}{Inhaltlich:} \usebeamerfont{text}\textit{Weitere Lerninhalte integrieren}
\begin{itemize}
	\item Weitere Inhalte, z.B. Rechte-Hand-Regel
	\item Weitere Experimente, z.B. Ablenkung eines Elektronenstrahles
\end{itemize}
\pause
\vskip 1em
\usebeamerfont{frametitle}\textcolor{blue}{Technische:} \usebeamerfont{text}\textit{Portierung für HoloLens 2}
\begin{itemize}
	\item Auflösung x4 pro Auge, Sichtfeld x2 (Fläche)
	\item Tragekomfort und Interaktion verbessert
\end{itemize}

\vspace{50px}
\end{frame}



\part{Diskussion}
\begin{frame}[fragile]{}
Vielen Dank für Ihre Aufmerksamkeit!

\vspace{1em}
\hspace{1em} Fragen?
\end{frame}
