\section{Einleitung}
\label{sec-1}
\fbox{
\parbox{\linewidth}{
	\textit{Ziel des Kapitels:}\\
	 Arbeit motivieren, an die Fragestellung heranführen und diese formulieren\\[6px]
	\textit{Inhalte:}
	\begin{itemize}
		\item MR kurz vorstellen, aktuelle Bedeutung hervorheben, HoloLens als aktuelles Device und Anhaltspunkt nutzen
		\item Anwendung in der Physik Lehre einleiten und motivieren
		\item Ziel, Aufgabenstellung und Struktur der Arbeit
	\end{itemize}
	
	\textit{Wichtige Literatur:}	
	\begin{itemize}
		\item Physics holo.lab learning experience: using smartglasses for augmented reality labwork to foster the concepts of heat conduction \cite{Strzys18}
		\item Augmenting Microsoft's HoloLens with vuforia tracking for neuronavigation \cite{Frantz18}
		\item HoloMuse: Enhancing Engagement with Archaeological Artifacts Through Gesture-Based Interaction with Holograms \cite{Pollalis17}
	\end{itemize}
}}


\subsection{Aufgabenstellung}
\label{sec-1-1}
Im Rahmen der Arbeit soll anhand der HoloLens untersucht werden, wie die Mixed Reality Technologie in der Physik-Lehre eingesetzt werden kann, um physikalische Inhalte zu vermitteln. Insbesondere soll betrachtet werden, wie physikalische Experimente mittels Mixed Reality Anwendungen durch zusätzliche Inhalte angereichert werden können.\\

\par
Dabei sind auf der einen Seite die technischen Möglichkeiten der HoloLens und des Mixed Reality Toolkits zu betrachten und in Zusammenhang mit dem Anwendungsfall zu bringen. Hier orientiert sich die Arbeit an bereits bestehenden Anwendungen wie z.B. in der Medizin. Auf der anderen Seite müssen, mit der Unterstützung durch Physiker, konkrete Anwendungsszenarien ausgearbeitet werden. Das beinhaltet eine Auswahl von geeigneten Experimenten, den darzustellenden Objekten und Informationen sowie das Zusammenspiel dieser mit dem aufgebauten Experiment, der Umgebung und den Nutzern.\\

\par
Für einen ausgewählten solchen Anwendungsfall soll eine Umsetzung mit der HoloLens konzipiert, designet und prototypisch implementiert werden.

\subsection{Aufbau der Arbeit}
\label{sec-1-2}
Die vorliegende Arbeit ist wie folgt aufgebaut. Kapitel \ref{sec-2} führt die Begriffe Augmented- und Mixed Reality ein und gibt einen kurzen Überblick über die Techniken. Kapitel \ref{sec-3} stellt die HoloLens mit ihren technischen Hintergründen vor und geht auf Designrelevante Aspekte ein. In Kapitel \ref{sec-4} wird der aktuelle Stand im Einsatz der HoloLens in der Physik und im Bereich Education allgemein beleuchtet und offene Probleme angesprochen. Den Entwicklungsprozess der Umsetzung betrachtet Kapitel \ref{sec-5}. Die erarbeiteten Konzepte diskutiert Kapitel \ref{sec-6}, gefolgt von einer Erörterung der prototypischen Umsetzung in Kapitel \ref{sec-7}. Abschließend fasst Kapitel \ref{sec-6} die Ergebnisse zusammen und zieht ein Fazit.