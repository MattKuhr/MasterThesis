\section{Einleitung}
\label{sec-1}
\fbox{
\parbox{\linewidth}{
	\textit{Ziel des Kapitels:}\\
	 Arbeit motivieren, an die Fragestellung heranführen und diese formulieren\\[6px]
	\textit{Inhalte:}
	\begin{itemize}
		\item MR kurz vorstellen, aktuelle Bedeutung hervorheben, HoloLens als aktuelles Device und Anhaltspunkt nutzen
		\item Anwendung in der Physik Lehre einleiten und motivieren
		\item Fragestellung, Ziel und Struktur der Arbeit
	\end{itemize}
	
	\textit{Wichtige Literatur:}	
	\begin{itemize}
		\item Physics holo.lab learning experience: using smartglasses for augmented reality labwork to foster the concepts of heat conduction \cite{Strzys18}
		\item Augmenting Microsoft's HoloLens with vuforia tracking for neuronavigation \cite{Frantz18}
		\item HoloMuse: Enhancing Engagement with Archaeological Artifacts Through Gesture-Based Interaction with Holograms \cite{Pollalis17}
	\end{itemize}
}}

\subsection{Motivation}
\label{sec-1-1}

\subsection{Fragestellung}
\label{sec-1-2}
Übergeordnete wissenschaftliche Fragestellung der Arbeit:
\begin{center}
	\textit{\textbf{Wie kann die HoloLens in der Physik-Lehre eingesetzt werden, um physikalische Zusammenhänge zu vermitteln?}}
\end{center}

Konkrete untergeordnete Frage:
\begin{center}
	\textit{\textbf{Wie kann die HoloLens in einem konkreten Anwendungsfall eines physikalischen Versuches genutzt werden?}}
\end{center}

\begin{comment}
\subsection{Aufgabenstellung}

Im Rahmen der Arbeit soll anhand der HoloLens untersucht werden, wie diese in der Physik-Lehre eingesetzt werden kann, um physikalische Inhalte zu vermitteln. Insbesondere soll betrachtet werden, wie physikalische Experimente mittels Mixed Reality Anwendungen durch zusätzliche Inhalte angereichert werden können.\\

\par
Dazu sind zunächst die technischen Möglichkeiten und Voraussetzungen der HoloLens zu betrachten und in Zusammenhang mit dem Anwendungsfall zu bringen. Weiterhin sind bestehende Ansätze im Einsatz von Mixed Reality Technologie in der Lehre, besonders in der Physik-Lehre, herauszuarbeiten und einzuordnen.

Davon ausgehend soll der Fragestellung anhand eines konkreten Beispiels nachgegangen werden. Für einen ausgewählten Versuchsaufbau sind die darzustellenden Objekte und Informationen sowie das Zusammenspiel dieser mit dem aufgebauten Experiment, der Umgebung und den Nutzern zu entwickeln. Für den ausgewählten Anwendungsfall soll eine Umsetzung mit der HoloLens konzipiert, designet und prototypisch implementiert werden.
\end{comment}

\subsection{Aufbau der Arbeit}
\label{sec-1-3}
Die vorliegende Arbeit ist wie folgt aufgebaut. Kapitel \ref{sec-2} erläutert notwendige Hintergrundinformationen zur Technik der HoloLens, Mixed Reality, dessen Anwendung im Bereich Education sowie die physikalischen Hintergründe des Versuches.  Kapitel \ref{sec-3} kristallisiert die Problemstellung und Anforderungen für den konkreten Anwendungsfall heraus. In Abschnitt \ref{sec-4} werden die gewählten Lösungsansätze vorgestellt, deren Umsetzung in Kapitel \ref{sec-5} erörtert wird. Kapitel \ref{sec-6} diskutiert diese dann im Rahmen der Fragestellung. Abschließend fasst Kapitel \ref{sec-7} die Ergebnisse zusammen und zieht ein Fazit.