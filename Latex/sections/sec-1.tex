\section{Einleitung}
\label{sec-1}
\fbox{
\parbox{\linewidth}{
	\textit{Ziel des Kapitels:}\\
	 Arbeit motivieren, an die Fragestellung heranführen und diese formulieren
}}

\subsection{Motivation}
\label{sec-1-1}
\begin{itemize}
	\item Augmented Reality ein aktuelles Innovationsgebiet
	\item Aktuelle Bedeutung durch Fortschritte bei Hardware und KI
	\item HoloLens erweitert Realität durch virtuelle 3D Objekte
	\item Einsatz in der Physik interessant
	\item Existierende Beispiele: Thermodynamik + HoloLens, Elektromagnetismus und WebCam-AR
	\item Versuch Helmholtzspulen aus 2. Beispiel aufgreifen
	\item Problem: Wie kann HoloLens genutzt werden, um Experiment durch virtuelle Darstellungen anzureichern?
	\item Motivation z.B. Einbinden von Informationen zu Versuchsaufbau, Visualisierung nicht sichtbarer physikalischer Größen und Vorgänge z.B. Magnetfeld und Stromfluss
\end{itemize}

\subsection{Fragestellung}
\label{sec-1-2}
Übergeordnete wissenschaftliche Fragestellung der Arbeit:
\begin{center}
	\textit{\textbf{Wie kann die HoloLens in der Physik-Lehre eingesetzt werden, um physikalische Zusammenhänge zu vermitteln?}}
\end{center}

Konkrete untergeordnete Frage:
\begin{center}
	\textit{\textbf{Wie kann die HoloLens in einem konkreten Anwendungsfall eines physikalischen Versuches genutzt werden?}}
\end{center}

\begin{comment}
\subsection{Aufgabenstellung}

Im Rahmen der Arbeit soll anhand der HoloLens untersucht werden, wie diese in der Physik-Lehre eingesetzt werden kann, um physikalische Inhalte zu vermitteln. Insbesondere soll betrachtet werden, wie physikalische Experimente mittels Mixed Reality Anwendungen durch zusätzliche Inhalte angereichert werden können.\\

\par
Dazu sind zunächst die technischen Möglichkeiten und Voraussetzungen der HoloLens zu betrachten und in Zusammenhang mit dem Anwendungsfall zu bringen. Weiterhin sind bestehende Ansätze im Einsatz von Mixed Reality Technologie in der Lehre, besonders in der Physik-Lehre, herauszuarbeiten und einzuordnen.

Davon ausgehend soll der Fragestellung anhand eines konkreten Beispiels nachgegangen werden. Für einen ausgewählten Versuchsaufbau sind die darzustellenden Objekte und Informationen sowie das Zusammenspiel dieser mit dem aufgebauten Experiment, der Umgebung und den Nutzern zu entwickeln. Für den ausgewählten Anwendungsfall soll eine Umsetzung mit der HoloLens konzipiert, designet und prototypisch implementiert werden.
\end{comment}

\subsection{Aufbau der Arbeit}
\label{sec-1-3}
Die vorliegende Arbeit ist wie folgt aufgebaut. Kapitel \ref{sec-2} erläutert notwendige Hintergrundinformationen zur Technik der HoloLens, Mixed Reality, dessen Anwendung im Bereich Education sowie die physikalischen Hintergründe des Versuches.  Kapitel \ref{sec-3} kristallisiert die Problemstellung und Anforderungen für den konkreten Anwendungsfall heraus. In Abschnitt \ref{sec-4} werden die gewählten Lösungsansätze vorgestellt, deren Umsetzung in Kapitel \ref{sec-5} erörtert wird. Kapitel \ref{sec-6} diskutiert diese dann im Rahmen der Fragestellung. Abschließend fasst Kapitel \ref{sec-7} die Ergebnisse zusammen und zieht ein Fazit.