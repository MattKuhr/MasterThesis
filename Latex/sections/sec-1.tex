\section{Einleitung}
\label{sec-1}
\fbox{
	\parbox{\linewidth}{
Ziele des Kapitels:		
\begin{itemize}
	\item MR kurz vorstellen, aktuelle Bedeutung hervorheben, HoloLens als aktuelles Device und Anhaltspunkt nutzen
	\item Anwendung in der Physik Lehre einleiten und motivieren
	\item Ziel, Aufgabenstellung und Struktur der Arbeit
\end{itemize}
}}


\subsection{Aufgabenstellung}
\label{sec-1-1}
Im Rahmen der Arbeit soll anhand der HoloLens untersucht werden, wie die Mixed Reality Technologie in der Physik-Lehre eingesetzt werden kann, um physikalische Inhalte zu vermitteln. Insbesondere soll betrachtet werden, wie physikalische Experimente mittels Mixed Reality Anwendungen durch zusätzliche Inhalte angereichert werden können.

Dabei sind auf der einen Seite die technischen Möglichkeiten der HoloLens und des Mixed Reality Toolkits zu betrachten und in Zusammenhang mit dem Anwendungsfall zu bringen. Hier orientiert sich die Arbeit an bereits bestehenden Anwendungen wie z.B. in der Medizin. Auf der anderen Seite müssen, mit der Unterstützung durch Physiker, konkrete Anwendungsszenarien ausgearbeitet werden. Das beinhaltet eine Auswahl von geeigneten Experimenten, den darzustellenden Objekten und Informationen sowie das Zusammenspiel dieser mit dem aufgebauten Experiment, der Umgebung und den Nutzern.

Für einen ausgewählten solchen Anwendungsfall soll eine Umsetzung mit der HoloLens konzipiert, designet und prototypisch implementiert werden.

\subsection{Aufbau der Arbeit}
\label{sec-1-2}

Die vorliegende Arbeit ist wie folgt aufgebaut. Kapitel \ref{sec-2} beleuchtet und motiviert den konkreten Anwendungsfall mit seinen spezifischen Anforderungen. Kapitel \ref{sec-3} stellt die technischen Hintergründe von Mixed Reality vor. In Kapitel \ref{sec-4} werden konkrete Konzepte zur Umsetzung diskutiert. Kapitel \ref{sec-5} verbindet die in den vorangegangenen Abschnitten erarbeiteten Erkenntnisse und diskutiert verschiedene Umsetzungsmöglichkeiten. Abschließend beschreibt Kapitel \ref{sec-6} Ergebnisse der Umsetzung und zieht ein Fazit.