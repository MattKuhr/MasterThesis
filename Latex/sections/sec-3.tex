\section{HoloLens}
\label{sec-3}
\fbox{
\parbox{\linewidth}{
	\textit{Ziel des Kapitels:}\\
	HoloLens und Mixed Reality Toolkit mit ihrer Technik und Interaktionsweise vorstellen und auf Implikationen für das Design eingehen.\\[6px]
	\textit{Inhalte:}	
	\begin{itemize}
		\item Übersicht über Technik der HoloLens, besonderer Fokus auf Designrelevanten Aspekten
		\item Interaktion und MRTK (kurz) vorstellen
		\item Einschränkungen und Designempfehlungen vorstellen
	\end{itemize}

	\textit{Wichtige Literatur:}	
	\begin{itemize}
		\item Mobile Augmented Reality Illustrations That Entertain and Inform: Design and Implementation Issues with the Hololens \cite{Zimmer17}
		\item Microsofts eigene Docs
		\item Ggf. Resourcen zu technischen Quellen bezüglich eingesetzter Techniken
	\end{itemize}
}}
	
\subsection{Die Technik}
\label{sec-3-1}
HMD mit see through display, 60hz upscale zu 240hz, jede Farbe einzeln sequenziell, pro Anwendungsframe also 3 Farbframes, inside out tracking mit IMU, 2x IR und Stereo Kamera, Head Movement Prediction, Standalone, CPU, GPU, Akkulaufzeit, Gewicht, Windows 10 Holographic, Entwicklung und Deployment mit Unity

\subsection{Interaktion}
\label{sec-3-2}
Gestensteuerung, Click, Hold-Click, Drag-Drop, Scale, Bloom usw. 1 und 2 Hand Gesten, Anwendungen können per API auf Gesten reagieren, Sprachsteuerung (Englisch), API für Spracherkennung

\subsection{Implikationen für Anwendungsdesign}
\label{sec-3-3}
Abstand, Geschwindigkeit und Größe der Objekte wichtig, Blickwinkel, Depth Cues, Drop Shadows, Occlusion, Einführung in Gesten, FoV, Cursor, Weiß - Rainbow, Dunkle Farben vermeiden, Performance, Klick-Feedback