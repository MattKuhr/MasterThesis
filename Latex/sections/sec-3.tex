\section{Problemstellung und Requirements}
\label{sec-3}
\fbox{
	\parbox{\linewidth}{
		\textit{Ziel des Kapitels:}\\
		Problemstellung formulieren und eingrenzen, Anforderungen herausarbeiten.\\[6px]
		\textit{Inhalte:}
		\begin{itemize}
			\item Problem
			\item Einschränkungen
			\item Anforderungen
		\end{itemize}
		
		\textit{(Optional) Literatur:}	
		%\begin{itemize}
		%\end{itemize}
}}

\subsection{Problemstellung}
\label{sec-3-1}
\begin{itemize}
	\item HoloLens Technik und Plattform mit eigenen Capabilities und Requirements
	\item Problem: Wie kann HoloLens beim Versuch Helmholtzspulen eingesetzt werden?
	\subitem Was soll dargestellt werden?
	\subitem Wie soll es dargestellt werden?
	\subitem Wie soll damit interagiert werden?
\end{itemize}

\subsection{Anforderungen}
\label{sec-3-2}
Anforderungen an die Anwendung
\begin{itemize}
	\item Homogenen Teil des Magnetfeldes schematisch darstellen
	\subitem der Erde
	\subitem der Spule
	\item Stromfluss darstellen
	\item Kräftevektoren darstellen
	\item Einzelne Darstellungen durch Interaktion ein/ausblenden
\end{itemize}

Technische Randbedingungen durch die Brille im Zusammenhang mit Anwendungsfall
\begin{itemize}
	\item Distanz, Größe und Geschwindigkeit der virtuellen Objekte, passt hier aufgrund Größe der Spulen
	\item Keine Störung der Brillen-Hardware (z.B. durch zusätzliches Infrarotlicht)
	\item Performance Limitierung, also z.B. keine Echtzeit-Berechnung der Magnetfelder
	\item FoV Limitierung muss beachtet werden
	\item Transparenz hängt von Farbe und Hintergrund ab
	\item ....
\end{itemize}

