\subsection{Der Helmholtzspulen Versuch}
\fbox{
	\parbox{\linewidth}{
		\textit{Ziel des Kapitels:}\\
		Anwendungsfall und Hintergrund vorstellen.
}}

\subsubsection{Physikalischer Hintergrund}
\begin{itemize}
	\item Homogenes Magnetfeld innerhalb Helmholtz-Spule
	\item Magnetfeld überlagerung, Vektoraddition
	\item Kompensation von Erdmagnetfeld
	\item ...
\end{itemize}

\subsubsection{Versuchsaufbau}
\begin{itemize}
	\item Helmholtz-Spulen, Windungszahl X, Abstand Y, Durchmesser Z
	\item Spannungsgenerator mit Spannung, Stromstärke, AC/DC
	\item Frei beweglicher Kompass im inneren der Spulen
	\item Rotation der Spulen sowie Spannungsanpassung durch Nutzer
	\item ... bis Kompass im inneren kein Magnetfeld mehr wahrnimmt
	\item ... dann ist das Magnetfeld der Erde genau kompensiert, d.h. genau so stark aber entgegengesetzt gerichtet zum Feld der Spulen
	\item Keine weitere Schutzausrüstung oder Messinstrumente notwendig
\end{itemize}