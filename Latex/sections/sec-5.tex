\section{Entwicklungsprozess}
\label{sec-5}
\fbox{
\parbox{\linewidth}{
	\textit{Ziel des Kapitels:}\\
	Den genutzten Entwicklungsprozess vorstellen und kurz anhand der Literatur begründen.\\[6px]
	\textit{Inhalte:}
	\begin{itemize}
		\item Vorgehen in Zusasmmenarbeit mit der Physik vorstellen
		\item Designprozess vorstellen
	\end{itemize}
	
	\textit{Wichtige Literatur:}	
	\begin{itemize}
		\item Envisioning Holograms \cite{Pell2017}
		\item Intuitive 3D Model Prototyping with Leap Motion and Microsoft HoloLens \cite{Jailungka18}
		\item Designing for Depth Perceptions in Augmented Reality \cite{Diaz17}
	\end{itemize}
}}

Zusammenarbeit mit Physikern hervorheben.

\subsection{Herausforderungen beim Design von MR Anwendungen}
Aufbauend auf Kap. 2, 3 und 4 die besonderen Herausforderungen des Designprozesses hervorheben. 

\subsection{Designprozess}
Gewähltens Vorgehen beschreiben und begründen.
Kein vertrauter Design-Space, deshalb Einführung über Beispielanwendungen, Anregungen über Vorschläge, Skizzen. Dann Verfeinerungsprozess. Orientierungshilfen aus \cite{Pell2017}. 

