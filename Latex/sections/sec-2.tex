\section{Hintergrund}
\label{sec-2}
\fbox{
	\parbox{\linewidth}{
		\textit{Ziel des Kapitels:}\\
		Hintergründe der Arbeit in 3 Abschnitten: HoloLens Technik, AR in Education und Physik Hintergründe.\\
}}\\


Physikalische Experimente durch virtuelle Darstellungen anzureichern und so besser und intuitiver verständlich zu machen, ist kein völlig neuer Ansatz. So stellen Strzys et. al. eine Anwendung mit der HoloLens im Bereich der Thermodynamik vor, bei der das gemessene Wärmeprofil eines erhitzten Metallstabes virtuell mit Hilfe der HoloLens auf den Stab gelegt wird \cite{Strzys17}. Und Buchau et. al. präsentieren eine Lösung, die unter anderem das Magnetfeld zweier Helmholtz-Spulen in das Echtzeitbild der Webcam zeichnet \cite{Buchau09}.\\

Um bestehende Ansätze einordnen und darauf aufbauen zu können ist es jedoch notwendig, zunächst. ...


Nicht zuletzt durch leistungsstärkere Hardware und Fortschritte im Bereich der künstlichen Intelligenz haben Augemented und Virtual Reality in den letzten Jahren an Bedeutung gewonnen. 