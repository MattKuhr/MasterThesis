	\section{Umsetzung und Fazit}
	\label{sec-6}
	DUMMY TEXT

	\begin{table}[htb]
		\centering
		\begin{tabular}{l l|r r r r r}
			 & & $k=5$ & $k=8$ & $k=10$ & $k=12$ & $k=14$\\
			\hline
			\multirow{4}{*}{\textit{Vertex-Cut}}			
			 & Cut-Size \textit{(opt)}& 5.689 & 6.254 & 6.355 & 6.480 & 6.543\\ 
			 & SoeD & 16.568 & 22.078 & 24.139 & 26.048 & 27.680\\ \cline{2-7}
			 & Cut-Size & 6.002 & 6.865 & 7.208 & 7.440 &7.776\\
			 & SoeD \textit{(opt)}& 16.101 & 20.748 & 22.778 & 24.563 & 26.169\\
		   \hline
		   \multirow{4}{*}{\textit{Edge-Cut}}
			& Cut-Size \textit{(opt)}& 33.067 & 42.039 & 46.243 & 49.775 & 52.441 \\
			& SoeD & 68.859 & 87.799 & 97.412 & 105.163 & 111.015\\ \cline{2-7}
			& Cut-Size & 33.647 & 42.163 & 46.714 & 50.161 & 53.002\\
			& SoeD \textit{(opt)}& 69.439 & 87.510 & 97.450 & 104.691 & 110.647\\
		\end{tabular}
	\caption{\label{tab:partition-stats} Die Ergebnisse mehrerer Zerlegungen mittels \textit{khmetis}. Welche Metrik als Zielfunktion diente, ist in Klammern gesetzt. Der Edge-Cut ist durchgehend deutlich schlechter für beide Metriken und Optimierungsziele. Partitioniert wurden 257.462 Knoten und 339.819 Hyperkanten.}
	\end{table}

	
	
	
	
	
	
	
	
	
	
	
	
	
	
	
	
	
	
	
	
	