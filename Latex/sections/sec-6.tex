\section{Konzepte für HoloLens in der Physik Lehre}
\label{sec-6}
\fbox{
\parbox{\linewidth}{
	\textit{Ziel des Kapitels:}\\
	Hier kommen die konzeptionellen Ergebnisse zum Einsatz der HoloLens in der Physik.\\[6px]
	\textit{Inhalte:}
	\textit{Hier kommen die konzeptionellen Ergebnisse zum Einsatz der HoloLens in der Physik}
	\begin{itemize}
		\item ToDo: Visualisierung vs. Simulation + Visualisierung
		\item ToDo: Verschiedene Konzepte anhand des Virtual Continuums vorstellen inkl. Beispiele
	\end{itemize}

	\textit{(Optional) Literatur:}	
	%\begin{itemize}
	%\end{itemize}
}}	

TODO

Idee: 
\subsection{Visualisierung und Simulation}
Visualisierung von Daten und Schemata
Simulation mit Visualisierung

\subsection{Anwendungsklassen}
in Anlehnung an Virtual Continuum
eigenständig vs angehängt vs eingebettet


	
	
	
	
	
	
	
	
	
	
	
	
	
	
	
	
	