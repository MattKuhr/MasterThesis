\section{Ergebnisse und Diskussion}
\label{sec-6}
\fbox{
\parbox{\linewidth}{
	\textit{Ziel des Kapitels:}\\
	Ergebnisse der Umsetzung vorstellen und auf die Fragestellung anwenden.\\[6px]
	\textit{Inhalte:}
	\textit{Ergebnisse}
	\begin{itemize}
		\item Vorstellung der Ergebnisse der Umsetzung
		\item Einordnung in die übergeordnete Fragestellung: Übertragung der Ergebnisse
	\end{itemize}
	
	\textit{(Optional) Literatur:}	
	%\begin{itemize}
	%\end{itemize}
}}	

\subsection{Ergebnisse im Bezug auf Problemstellung}
\begin{itemize}
	\item Erweiterung des Versuches um Magnetfeld, Stromfluss, Kräftevektoren
	\item Nutzer hat die Hände frei und kann Spulen bewegen und vor allem drehen
	\item Anwendung mit intrinsic contextuality, da Darstellungen eingebettet sind
	\item Anwendung zeigt nicht sichtbare, reale physikalische Zusammenhänge des Experimentes
	\item Depth Cues zur besseren Tiefenwahrnehmungen genutzt
	\item Empfohlene Constraints berücksichtigt:
	\subitem Distanz, Größe, Geschwindigkeit, Blickwinkel
	\subitem Framerate
	\item Nicht umgesetzt werden konnte eine Einführung in die Gestensteuerung
	\item ...
\end{itemize}

\subsection{Diskussion}
\begin{itemize}
	\item Anwendung kann erweitert werden, z.B. um Ablenkung von Elektronen in der Spule
	\item 3D Darstellungen der HoloLens genutzt für Magnetfeld der Spulen, auch für andere Magnetfelder anwendbar
	\item Kräftevektoren ebenfalls übertragbar
	\item Einbettung von Simulation denkbar und sinnvoll, denn das erlaubt den Nutzern What-If Szenarien durchzuspielen, sehr wichtig zum Lernen
	\item ...
\end{itemize}
	
	
	
	
	
	
	
	
	
	
	
	
	
	
	