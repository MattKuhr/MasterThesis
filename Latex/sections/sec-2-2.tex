\subsection{Mixed Reality in der Lehre}
\label{sec-2-2}

\fbox{
\parbox{\linewidth}{
	\textit{Ziel des Kapitels:}\\
	State of the Art in der Überschneidung mit dem Education Bereich vorstellen.\\[6px]
	\textit{Inhalte:}
	\begin{itemize}
		\item Aktueller Einsatz von MR, insb. der HoloLens, in der Physik
		\item Einsatz von MR in Lehre allgemein, nur relevante Aspekte
	\end{itemize}

	\textit{Wichtige Literatur:}	
	\begin{itemize}
		\item Physics holo.lab learning experience: using smartglasses for augmented reality labwork to foster the concepts of heat conduction \cite{Strzys18}
		\item Using Augmented Reality for Teaching Physics \cite{Techakosit15}
		\item HolOsci: Hololens Augmented Reality Oscilloscope Based Support for Debugging Electronics Circuits \cite{Javaheri18}
		\item PhET: Simulations that enhance learning \cite{Wieman08}
	\end{itemize}
}}	

\subsubsection{In der Physik}
Buchau, Strzys, Javaheri, Techakosti vorstellen

\subsubsection{In anderen Bereichen}
Beispiele vorstellen, Paper zu positiven Effekten von AR zitieren als Legitimation, Einsatz begründen, Beispiele vorstellen, insb. mit der HoloLens

