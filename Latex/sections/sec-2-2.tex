\subsection{Mixed Reality in der Lehre}
\label{sec-2-2}
\fbox{
\parbox{\linewidth}{
	\textit{Ziel des Kapitels:}\\
	State of the Art in der Überschneidung mit dem Education Bereich vorstellen.
}}	

\begin{itemize}
	\item Bildung ist wichtiges Anwendungsgebiet für MR im Allgemeinen und AR im Besonderen
	\item Ergänzung des realen ohne völlige Abschottung im Fall von AR
	\item Direkte Verknüpfung (räumlich-zeitlich) für Lerneffekt vermutlich positiv \cite{Akcayir17}, \cite{Chen2017}, \cite{Knierim18}
\end{itemize}

\subsubsection{In der Physik}
Buchau \cite{Buchau09}, Strzys \cite{Strzys18}, Javaheri \cite{Javaheri18}, Techakosti \cite{Techakosit15} vorstellen. Weitere Arbeiten erwähnen.\\
Einordnung in Virtual Continuum und Anwendungsgebiet in der Physik.
\begin{itemize}
	\item Buchau links im AR Bereich, Magnetismus und Elektromagnetismus im Anwendungsbereich
	\item Strzys weiter rechts im AR Bereich, Thermodynamik im Anwendungsbereich
	\item Javaheri mittig im AR Bereich, Elektronik und Schaltungen im Anwendungsbereich
	\item ...
	\item ...
\end{itemize}

\subsubsection{In anderen Bereichen}
\begin{itemize}
	\item Beispiel HoloMuse, Insight Heart, Galaxy Explorer 
\end{itemize}

Fazit: So und so wird MR und HoloLens aktuell eingesetzt

