\section{Mixed Reality in der Lehre}
\label{sec-4}
\fbox{
\parbox{\linewidth}{
	\textit{Ziel des Kapitels:}\\
	State of the Art in der Überschneidung mit dem Education Bereich vorstellen.\\[6px]
	\textit{Inhalte:}
	\begin{itemize}
		\item Aktueller Einsatz der HoloLens insb. In der Physik, aber auch in anderen Bereichen
		\item Einsatz von AR in Lehre allgemein, nur relevante Aspekte
		\item Offene Probleme und Fragestellungen aufzeigen, insb. die aktuell fehlende Einbettung von Darstellungen in reale Objekte und das Potential für Simulationen
	\end{itemize}

	\textit{Wichtige Literatur:}	
	\begin{itemize}
		\item Physics holo.lab learning experience: using smartglasses for augmented reality labwork to foster the concepts of heat conduction \cite{Strzys18}
		\item Using Augmented Reality for Teaching Physics \cite{Techakosit15}
		\item HolOsci: Hololens Augmented Reality Oscilloscope Based Support for Debugging Electronics Circuits \cite{Javaheri18}
		\item PhET: Simulations that enhance learning \cite{Wieman08}
	\end{itemize}
}}	

\subsection{AR im Education Bereich}
Beispiele vorstellen, Paper zu positiven Effekten von AR zitieren als Legitimation, Einsatz begründen, Beispiele vorstellen, insb. mit der HoloLens

\subsection{Einsatz in der Physik}
Strzys, Javaheri, Techakosti vorstellen

\subsection{Offenes Potential in der Physik}
fehlende Integration von Einbettung und Interaktion der Darstellungen sowie keine Simulation